% Options for packages loaded elsewhere
\PassOptionsToPackage{unicode}{hyperref}
\PassOptionsToPackage{hyphens}{url}
\PassOptionsToPackage{dvipsnames,svgnames,x11names}{xcolor}
%
\documentclass[
  12pt,
  letterpaper,
]{scrartcl}

\usepackage{amsmath,amssymb}
\usepackage{iftex}
\ifPDFTeX
  \usepackage[T1]{fontenc}
  \usepackage[utf8]{inputenc}
  \usepackage{textcomp} % provide euro and other symbols
\else % if luatex or xetex
  \usepackage{unicode-math}
  \defaultfontfeatures{Scale=MatchLowercase}
  \defaultfontfeatures[\rmfamily]{Ligatures=TeX,Scale=1}
\fi
\usepackage{lmodern}
\ifPDFTeX\else  
    % xetex/luatex font selection
\fi
% Use upquote if available, for straight quotes in verbatim environments
\IfFileExists{upquote.sty}{\usepackage{upquote}}{}
\IfFileExists{microtype.sty}{% use microtype if available
  \usepackage[protrusion = true]{microtype}
  \UseMicrotypeSet[protrusion]{basicmath} % disable protrusion for tt fonts
}{}
\makeatletter
\@ifundefined{KOMAClassName}{% if non-KOMA class
  \IfFileExists{parskip.sty}{%
    \usepackage{parskip}
  }{% else
    \setlength{\parindent}{0pt}
    \setlength{\parskip}{6pt plus 2pt minus 1pt}}
}{% if KOMA class
  \KOMAoptions{parskip=half}}
\makeatother
\usepackage{xcolor}
\usepackage[margin = 1in]{geometry}
\setlength{\emergencystretch}{3em} % prevent overfull lines
\setcounter{secnumdepth}{2}
% Make \paragraph and \subparagraph free-standing
\makeatletter
\ifx\paragraph\undefined\else
  \let\oldparagraph\paragraph
  \renewcommand{\paragraph}{
    \@ifstar
      \xxxParagraphStar
      \xxxParagraphNoStar
  }
  \newcommand{\xxxParagraphStar}[1]{\oldparagraph*{#1}\mbox{}}
  \newcommand{\xxxParagraphNoStar}[1]{\oldparagraph{#1}\mbox{}}
\fi
\ifx\subparagraph\undefined\else
  \let\oldsubparagraph\subparagraph
  \renewcommand{\subparagraph}{
    \@ifstar
      \xxxSubParagraphStar
      \xxxSubParagraphNoStar
  }
  \newcommand{\xxxSubParagraphStar}[1]{\oldsubparagraph*{#1}\mbox{}}
  \newcommand{\xxxSubParagraphNoStar}[1]{\oldsubparagraph{#1}\mbox{}}
\fi
\makeatother

\providecommand{\tightlist}{%
  \setlength{\itemsep}{0pt}\setlength{\parskip}{0pt}}\usepackage{longtable,booktabs,array}
\usepackage{calc} % for calculating minipage widths
% Correct order of tables after \paragraph or \subparagraph
\usepackage{etoolbox}
\makeatletter
\patchcmd\longtable{\par}{\if@noskipsec\mbox{}\fi\par}{}{}
\makeatother
% Allow footnotes in longtable head/foot
\IfFileExists{footnotehyper.sty}{\usepackage{footnotehyper}}{\usepackage{footnote}}
\makesavenoteenv{longtable}
\usepackage{graphicx}
\makeatletter
\def\maxwidth{\ifdim\Gin@nat@width>\linewidth\linewidth\else\Gin@nat@width\fi}
\def\maxheight{\ifdim\Gin@nat@height>\textheight\textheight\else\Gin@nat@height\fi}
\makeatother
% Scale images if necessary, so that they will not overflow the page
% margins by default, and it is still possible to overwrite the defaults
% using explicit options in \includegraphics[width, height, ...]{}
\setkeys{Gin}{width=\maxwidth,height=\maxheight,keepaspectratio}
% Set default figure placement to htbp
\makeatletter
\def\fps@figure{htbp}
\makeatother

% NOTE: KOMAscript settings for header and footer
\usepackage[headsepline = 1.5pt, footsepline = 1.5pt]{scrlayer-scrpage}
\setkomafont{pagenumber}{\bfseries}
\setkomafont{pagehead}{\bfseries}
\setkomafont{pagefoot}{\itshape}
\lohead*{Exam 1 Study Guide}
\rohead*{\pagemark}
\cofoot*{"Scientists have become the bearers of the torch of discovery in our quest for knowledge" - Stephen Hawking}

% NOTE: Color settings
\usepackage{color}
\definecolor{gvblue}{HTML}{0032A0} % NOTE: Setting a new color
\renewcommand\thesection{\color{gvblue}\arabic{section}} % NOTE: Change section number color

% NOTE: Symbols, Fonts, and typesetting
\usepackage{bbding} % NOTE: For checkmark with \Checkmark
\usepackage{mathtools, amsthm, amssymb, amsfonts}
\usepackage{lmodern}
\usepackage[utf8]{inputenc}

% NOTE: Double-spacing and indents
% \usepackage{setspace} % NOTE: Allows for specifying spaces
% \usepackage{indentfirst} % NOTE: Indent first paragraph

\usepackage{multicol}
\makeatletter
\@ifpackageloaded{caption}{}{\usepackage{caption}}
\AtBeginDocument{%
\ifdefined\contentsname
  \renewcommand*\contentsname{Table of contents}
\else
  \newcommand\contentsname{Table of contents}
\fi
\ifdefined\listfigurename
  \renewcommand*\listfigurename{List of Figures}
\else
  \newcommand\listfigurename{List of Figures}
\fi
\ifdefined\listtablename
  \renewcommand*\listtablename{List of Tables}
\else
  \newcommand\listtablename{List of Tables}
\fi
\ifdefined\figurename
  \renewcommand*\figurename{Figure}
\else
  \newcommand\figurename{Figure}
\fi
\ifdefined\tablename
  \renewcommand*\tablename{Table}
\else
  \newcommand\tablename{Table}
\fi
}
\@ifpackageloaded{float}{}{\usepackage{float}}
\floatstyle{ruled}
\@ifundefined{c@chapter}{\newfloat{codelisting}{h}{lop}}{\newfloat{codelisting}{h}{lop}[chapter]}
\floatname{codelisting}{Listing}
\newcommand*\listoflistings{\listof{codelisting}{List of Listings}}
\makeatother
\makeatletter
\makeatother
\makeatletter
\@ifpackageloaded{caption}{}{\usepackage{caption}}
\@ifpackageloaded{subcaption}{}{\usepackage{subcaption}}
\makeatother

\usepackage{hyphenat}
\usepackage{ifthen}
\usepackage{calc}
\usepackage{calculator}



\usepackage{graphicx}
\usepackage{geometry}
\usepackage{afterpage}
\usepackage{tikz}
\usetikzlibrary{calc}
\usetikzlibrary{fadings}
\usepackage[pagecolor=none]{pagecolor}


% Set the titlepage font families







% Set the coverpage font families


\ifLuaTeX
\usepackage[bidi=basic]{babel}
\else
\usepackage[bidi=default]{babel}
\fi
\babelprovide[main,import]{american}
% get rid of language-specific shorthands (see #6817):
\let\LanguageShortHands\languageshorthands
\def\languageshorthands#1{}
\ifLuaTeX
  \usepackage{selnolig}  % disable illegal ligatures
\fi
\usepackage{bookmark}

\IfFileExists{xurl.sty}{\usepackage{xurl}}{} % add URL line breaks if available
\urlstyle{same} % disable monospaced font for URLs
\hypersetup{
  pdftitle={PSY-300 Exam 1 Study Guide},
  pdfauthor={Quinton Quagliano, M.S.},
  pdflang={en-US},
  pdfsubject={PSY-300 Exam 1 Study Guide},
  pdfkeywords={Exam Study Guide},
  colorlinks=true,
  linkcolor={gvblue},
  filecolor={Maroon},
  citecolor={gvblue},
  urlcolor={gvblue},
  pdfcreator={LaTeX via pandoc}}


\title{Exam 1 Study Guide - Fall 2024}
\usepackage{etoolbox}
\makeatletter
\providecommand{\subtitle}[1]{% add subtitle to \maketitle
  \apptocmd{\@title}{\par {\large #1 \par}}{}{}
}
\makeatother
\subtitle{PSY-300: Research Methods in Psychology}
\author{Quinton Quagliano, M.S.}
\date{Friday, September 27, 2024}

\begin{document}
%%%%% begin titlepage extension code


\begin{titlepage}

%%% TITLE PAGE START

% Set up alignment commands
%Page
\newcommand{\titlepagepagealign}{
\ifthenelse{\equal{center}{right}}{\raggedleft}{}
\ifthenelse{\equal{center}{center}}{\centering}{}
\ifthenelse{\equal{center}{left}}{\raggedright}{}
}


\newcommand{\titleandsubtitle}{
% Title and subtitle
{{\huge{\bfseries{\nohyphens{Exam 1 Study Guide - Fall 2024}}}}\par
}%

\vspace{\betweentitlesubtitle}
{
{\Large{\nohyphens{PSY-300: Research Methods in Psychology}}}\par
}}
\newcommand{\titlepagetitleblock}{
\newcommand{\HRule}{\rule{\linewidth}{0.5mm}} 

\HRule\\[0.4cm]

\titleandsubtitle

\HRule\\
}
\newcommand{\authorstyle}[1]{{\small{#1}}}

\newcommand{\affiliationstyle}[1]{{\small{#1}}}

\newcommand{\titlepageauthorblock}{
\newlength{\miniA}
\setlength{\miniA}{0pt}
\newlength{\namelen}
\settowidth{\namelen}{Quinton Quagliano,
M.S.}\setlength{\miniA}{\maxof{\miniA}{\namelen}}
\setlength{\miniA}{\miniA+0.05\textwidth}
\newlength{\miniB}
\setlength{\miniB}{0.99\textwidth - \miniA}
\begin{minipage}{\miniA}
\begin{flushleft}
{\authorstyle{Quinton Quagliano, M.S.}}
\end{flushleft}
\end{minipage}
\begin{minipage}{\miniB}
\begin{flushright}
{\affiliationstyle{Department of Psychology
\\}}
\end{flushright}
\end{minipage}}

\newcommand{\titlepageaffiliationblock}{}
\newcommand{\headerstyled}{%
{\textsc{\LARGE{}}}
}
\newcommand{\footerstyled}{%
{}
}
\newcommand{\datestyled}{%
{\large{Friday, September 27, 2024}}
}


\newcommand{\titlepageheaderblock}{\headerstyled}

\newcommand{\titlepagefooterblock}{
\footerstyled
}

\newcommand{\titlepagedateblock}{
\datestyled
}

%set up blocks so user can specify order
\newcommand{\titleblock}{\newlength{\betweentitlesubtitle}
\setlength{\betweentitlesubtitle}{\baselineskip}
{

{\titlepagetitleblock}
}

\vspace{1.5cm}
}

\newcommand{\authorblock}{{\titlepageauthorblock}

\vspace{1.5cm}
}

\newcommand{\affiliationblock}{{\titlepageaffiliationblock}

\vspace{0pt}
}

\newcommand{\logoblock}{{\includegraphics[width=250pt]{gvlogo.png}}

\vspace{2\baselineskip}
}

\newcommand{\footerblock}{}

\newcommand{\dateblock}{{\titlepagedateblock}

\vspace{0pt}
}

\newcommand{\headerblock}{}

\thispagestyle{empty} % no page numbers on titlepages


\newlength{\minipagewidth}
\setlength{\minipagewidth}{\textwidth}
\raggedright % single minipage
% [position of box][box height][inner position]{width}
% [s] means stretch out vertically; assuming there is a vfill
\begin{minipage}[b][\textheight][s]{\minipagewidth}
\titlepagepagealign
\headerblock

\logoblock

\titleblock

\authorblock
\par

\end{minipage}\ifthenelse{\equal{}{right} \OR \equal{}{leftright} }{
\hspace{\B}
\vrulecode}{}
\clearpage
%%% TITLE PAGE END
\end{titlepage}
\setcounter{page}{1}

%%%%% end titlepage extension code

% NOTE: Better hold for LaTeX tables
\floatplacement{table}{H}

% NOTE: Set KOMAscript headings
\pagestyle{scrheadings}

% NOTE: Double-spacing and indents
% \doublespacing
\frenchspacing % NOTE: Only single space after period
% \setlength{\parindent}{20pt} % NOTE: Amount of spacing in indent

\renewcommand*\contentsname{Table of Contents}
{
\hypersetup{linkcolor=}
\setcounter{tocdepth}{2}
\tableofcontents
}

\newpage{}

\section{Exam 1 Format \& Structure}\label{sec-format}

\emph{From the syllabus:}

\begin{itemize}
\item
  Each exam is 50 multiple-choice questions, 2 points for each question.
\item
  Exams will be taken at the start of the class period, but after the
  quiz review for the prior week. They will be paper forms (i.e., not on
  Blackboard).
\item
  Exams are timed, 113 minutes total (previously was 75 minutes).
\item
  Exams are \textbf{not} open-note, open-book, or collaborative. You are
  \textbf{not} permitted to use any form of assistance to aid you during
  the tests. Do not discuss the test with other students, even after it
  has concluded.
\item
  Any indication of academic dishonesty or ``cheating'' will be
  investigated thoroughly and will result in an automatic 0 on the exam
  for offenders
\item
  Exams will be ended early if all students are clearly finished and
  content with their answers.
\item
  Exams will be graded promptly and reviewed the following week.
\item
  Exams will contain content from the entire unit, from lectures,
  readings, and other class activities. This will include content from
  weeks/chapter 1 though 6.
\item
  Exams will not be purely vocabulary-based, students should have a
  solid understanding of applications of concepts, ideas, and theories.
\end{itemize}

\section{Using This Study Guide \& Other Resources}\label{sec-using}

This study guide is meant to help get students started with guided
questions and tasks that will aid performance on Exam 1. It is laid out
as examples and open-ended questions to provoke thought on the most
pressing questions of each chapter. You may consider using the slides,
recorded lectures, and textbook to help you address each part. However,
there are other valuable resources you may use to help you prepare:

\begin{itemize}
\item
  Review the textbook and professor learning objectives throughout the
  slides and chapters.
\item
  Use the questions (``Check Your Understanding'') throughout the
  chapters to quiz yourself and use the lengthy review sections at the
  end of each chapter to get a nice sampling of practice activities.
\item
  Use the results and answers from the weekly quizzes to identify areas
  of need for studying.
\item
  Make flashcards for important vocabulary.
\item
  Re-watch recorded lectures to see if you missed any classes and to
  ensure you haven't missed any critical content.
\item
  If you exhaust your other options for reviewing content, please feel
  free to ask me questions as well.
\end{itemize}

If you wish to perform your best, you should use a combination of all of
these available resources to help you prepare. Furthermore, I make no
guarantee that this study guide will contain \emph{all} the information
on the exam - it is the student's responsibility to review all materials
related to the first 6 weeks of content.

\section{Chapter/Week 1}\label{sec-ch-1}

\begin{itemize}
\item
  What is a research \textbf{producer}? What is a job title this person
  is likely to have? What sorts of activities is this person likely to
  perform?
\item
  What is a research \textbf{consumer}? What is a job title this person
  is likely to have? What sorts of activities is this person likely to
  perform?
\item
  Describe the procedures, findings, and structure of Harlow's monkey
  study to investigate cupboard theory vs.~contact comfort theory.
\item
  Describe the concept of \textbf{Empiricism} and its relationship to
  the \textbf{Theory-Hypothesis-Data Cycle.} Define and explain each
  individual part of the cycle.
\item
  How are Empiricism, the process of \textbf{reproducing} scientific
  studies, and the \textbf{self-correcting} nature of science related?
\item
  What are \textbf{Merton's 4 Scientific Norms}? Be able to give
  definitions of each and examples of behaviors that support these
  norms.
\item
  Why don't we use the word ``prove'' in science, and why do we prefer
  using the \textbf{weight of evidence} to describe support for a claim?
  Write out an example of what an improper claim statement (i.e., one
  being too certain) may look like, and then give a ``fixed'' version of
  that statement.
\item
  What are the 3 contexts/types of research? Give an example of a study
  that would be described as each type.
\item
  Compare and contrast the methods and trust-worthiness of
  \textbf{scientific journalism} and \textbf{scientific articles
  published in peer-reviewed journals.} See if you can find an example
  of both types, or better, find a piece of journalism and then the
  original article the journalism was based upon.
\item
  What citation style do we use in this course? What are the core,
  important components of a full citation (Like that found in a
  ``References'' section)? Looking at the information of an article, try
  to type out the correct formatted citation for it. What are some
  shortcuts in Google Scholar or scientific databases to get the
  citation information?
\end{itemize}

\section{Chapter/Week 2}\label{sec-ch-2}

\begin{itemize}
\item
  What are the 4 sources of knowledge or information we can draw from?
  Which of these is the most sound?
\item
  Why do we consider research to be superior to \textbf{authority}?
  Provide 2 examples of faults in trusting authority.
\item
  Why do we consider research to be superior to \textbf{personal
  experience}? Provide 2 examples of faults in relying on personal
  experience.
\item
  Why do we consider research to be superior to \textbf{intuition}?
  Provide 2 examples of biases in intuition.
\item
  Describe the dangers of \textbf{confounds} and having no
  \textbf{comparison groups.} What issues do these introduce? Describe
  an example of a confound.
\item
  What does is mean that research is \textbf{probabilistic}? How does
  this affect how we apply research findings to individual people?
\item
  Describe the 6 biases we discussed in intuition. Write an example of
  each bias.
\item
  What is the difference between an \textbf{original empirical article,}
  a \textbf{literature review}, and a \textbf{meta-analysis?} Try to
  find an example of each of these using the techniques we learned to
  find research.
\item
  What type of search engine should I use in order to find
  \emph{peer-reviewed} journal articles? Give two examples of
  appropriate tools. Find a scientific article that is specifically
  about traumatic brain injury and recovery.
\item
  What are the procedures for fully inspecting a research article before
  using it?
\item
  Why are books in science generally less valued as a source, compared
  to scientific journal articles? Connect this back to criticisms of
  authority.
\end{itemize}

\section{Chapter/Week 3}\label{sec-ch-3}

\begin{itemize}
\item
  Define a \textbf{variable} and a \textbf{constant}. Give an example of
  each in a research setting.
\item
  What are the 3 different \textbf{scales of measurement} for variables?
  Give an example of a variable of each type.
\item
  What is the difference between a \textbf{measured} and
  \textbf{manipulated} variables? What are some reasons a measure may
  have to measured, and could not be manipulated?
\item
  Compare and contrast \textbf{construct} and \textbf{operational}
  variables. What are examples of both of these? For the construct of
  anxiety, propose a simple operational measure.
\item
  What are the 3 types of \textbf{claims} we can make? Give one example
  of each type and a few keywords that signal each type of claim. Which
  of these claims requires the ``hardest'' burden to investigate?
\item
  What statistical and graphical methods do we have for presenting
  information about associative claims? How does one interpret each of
  these?
\item
  What are the three specified criteria that must be met that allow us
  to determine causation between two or more things?
\item
  What are the 4 \textbf{claim validities} for examining scientific
  claims? For each validity, give one example of a detriment and one
  example of a benefit. Which of the validities is only really
  applicable to causal claims?
\item
  Describe some primary issues in the conclusions of Mak et al.~(2023;
  the article we walked through in class). With each one of the
  validities, how well does the article meet them?
\end{itemize}

\section{Chapter/Week 4}\label{sec-ch-4}

\begin{itemize}
\item
  In the case of Prof.~Gino of Harvard Business School, what were the
  primary accusations levied against her, and what was her response?
  What were the consequences of her actions?
\item
  Be able to explain the general structure and goals of the
  \textbf{Tuskegee Syphilis Study} and the \textbf{Milgram Obedience
  Studies}.
\item
  Describe the questionable/bad ethics of the Tuskegee Syphilis Study
  and the Milgram Obedience Studies. In what way did these studies cause
  undue harm or fail the participants? Connect these failing to specific
  principles or guidelines in the \textbf{APA Code of Ethics}.
\item
  Enumerate the 3 principles of the \textbf{Belmont Report} and the 5
  principles of the APA Code of Ethics. Be able to define each of these
  \emph{in detail.}
\item
  Describe the relationship between \textbf{informed consent},
  \textbf{deception}, and \textbf{debriefing}. In what cases would
  deception be allowed?
\item
  What are the governing bodies for ensuring ethical research procedures
  at any institution? In what ways can they influence and regulate
  ongoing projects?
\item
  Explain the difference between the two types of \textbf{research
  misconduct} - which type was Prof.~Gino (credibly) accused of?
\item
  The first chapter mentioned \textbf{``Pre-registration''} of
  hypotheses - which of the APA Code of Ethics does this best relate to?
\item
  Describe a situation where ethics and study validity may be at odds or
  in conflict. What is an example of a study that would be
  scientifically valid, but not ethical?
\end{itemize}

\section{Chapter/Week 5}\label{sec-ch-5}

\begin{itemize}
\item
  What type of variable requires a \textbf{conceptual definition?}
\item
  What are the 3 types or mediums of psychological measures? Provide an
  example of each type and describe what these might look like in
  practice.
\item
  What is the relationship between measurement \textbf{reliability} and
  \textbf{validity}? What claim validity do these both fall under?
\item
  Describe the concept of reliability: what is a good synonym for it?
  Provide a real life example of something ``unreliable'' in the
  measurement sense. What are the 3 types of reliability discussed in
  class?
\item
  What is the most common graphical and statistical method to assess
  reliability? How does one interpret each of these graphs and
  statistics? (You do not need to know how to calculate statistics, but
  you should know what certain values would imply)
\item
  Define measurement validity - what is a good synonym for it? What are
  the 5 types of measurement validity we discussed in class? Which of
  these are the ``weakest'' or the most subjective?
\item
  What are the two methods by which to investigate \textbf{criterion
  validity}? What sorts of analyses do we use in the case of either
  type?
\item
  Using the visual analogy of targets (as presented in class), what
  would a reliable, but invalid tool look like?
\item
  By what methods can we find established measures for common
  psychological constructs? How is this similar to searching for
  scientific literature?
\end{itemize}

\section{Chapter/Week 6}\label{sec-ch-6}

\begin{itemize}
\item
  When we use the terms \textbf{survey} or \textbf{pool} what
  type/medium are we referring to? What are the roughly 4 types of
  questions we talked about in class? Write an example of each question
  type.
\item
  What are the 4 threats we discussed in how questions are written? What
  are the available solutions to address each one of these threats?
\item
  Describe what a \textbf{response set} is and why it is undesirable in
  participant responses. What are the 3 to 4 response sets we talked
  about, and how can we attempt to prevent them in our research?
\item
  Describe the concerns in long-term memory and peoples' ability to
  introspect on \emph{why} they do something, which could affect our
  survey results?
\item
  Describe the difference between an \textbf{observational measure} and
  a \textbf{self-report} one. Why might a psychologist see one as more
  ``objective'' than the other?
\item
  What are the 3 major concerns to measurement validity in observational
  measures? Give an example of each one of these occurring and how we
  might solve each.
\item
  In what ways may we choose to use deception in observation tools to
  enhance validity?
\end{itemize}

\section{Inter-chapter Questions}\label{sec-inter-chapter}

These questions are meant to compound different ideas from the lecture
and synthesize what we have learned so far. They are difficult, but also
realistically how complex research normally is.

\begin{itemize}
\item
  I want to run a study observing people at my local mall and their
  shopping and impulsive buying tendencies - what sort of consent do I
  need from the participants? If I want to give them a self-report form,
  and interact with them directly, does that change how I need to get
  consent? Is there an existing measure for understanding shopping
  tendencies I can find somewhere? If I need to make a new measure, what
  sorts of questions do I put on it?
\item
  I want to publish research that I did as (hypothetically) a
  high-school student. I didn't have any supervision but I planned and
  ran it all just fine, it was a study about giving supplements to
  people and seeing their reactions over time. Now I can't get any
  journal to publish it, only because I am a high school student and
  don't have credentials. What scientific principals may be violated
  here, both by journals and by the student?
\end{itemize}




\end{document}
