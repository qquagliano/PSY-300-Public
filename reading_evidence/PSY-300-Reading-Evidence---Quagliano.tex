% Options for packages loaded elsewhere
\PassOptionsToPackage{unicode}{hyperref}
\PassOptionsToPackage{hyphens}{url}
\PassOptionsToPackage{dvipsnames,svgnames,x11names}{xcolor}
%
\documentclass[
  12pt,
  letterpaper,
]{scrartcl}

\usepackage{amsmath,amssymb}
\usepackage{iftex}
\ifPDFTeX
  \usepackage[T1]{fontenc}
  \usepackage[utf8]{inputenc}
  \usepackage{textcomp} % provide euro and other symbols
\else % if luatex or xetex
  \usepackage{unicode-math}
  \defaultfontfeatures{Scale=MatchLowercase}
  \defaultfontfeatures[\rmfamily]{Ligatures=TeX,Scale=1}
\fi
\usepackage{lmodern}
\ifPDFTeX\else  
    % xetex/luatex font selection
\fi
% Use upquote if available, for straight quotes in verbatim environments
\IfFileExists{upquote.sty}{\usepackage{upquote}}{}
\IfFileExists{microtype.sty}{% use microtype if available
  \usepackage[protrusion = true]{microtype}
  \UseMicrotypeSet[protrusion]{basicmath} % disable protrusion for tt fonts
}{}
\makeatletter
\@ifundefined{KOMAClassName}{% if non-KOMA class
  \IfFileExists{parskip.sty}{%
    \usepackage{parskip}
  }{% else
    \setlength{\parindent}{0pt}
    \setlength{\parskip}{6pt plus 2pt minus 1pt}}
}{% if KOMA class
  \KOMAoptions{parskip=half}}
\makeatother
\usepackage{xcolor}
\usepackage[margin = 1in]{geometry}
\setlength{\emergencystretch}{3em} % prevent overfull lines
\setcounter{secnumdepth}{2}
% Make \paragraph and \subparagraph free-standing
\makeatletter
\ifx\paragraph\undefined\else
  \let\oldparagraph\paragraph
  \renewcommand{\paragraph}{
    \@ifstar
      \xxxParagraphStar
      \xxxParagraphNoStar
  }
  \newcommand{\xxxParagraphStar}[1]{\oldparagraph*{#1}\mbox{}}
  \newcommand{\xxxParagraphNoStar}[1]{\oldparagraph{#1}\mbox{}}
\fi
\ifx\subparagraph\undefined\else
  \let\oldsubparagraph\subparagraph
  \renewcommand{\subparagraph}{
    \@ifstar
      \xxxSubParagraphStar
      \xxxSubParagraphNoStar
  }
  \newcommand{\xxxSubParagraphStar}[1]{\oldsubparagraph*{#1}\mbox{}}
  \newcommand{\xxxSubParagraphNoStar}[1]{\oldsubparagraph{#1}\mbox{}}
\fi
\makeatother

\providecommand{\tightlist}{%
  \setlength{\itemsep}{0pt}\setlength{\parskip}{0pt}}\usepackage{longtable,booktabs,array}
\usepackage{calc} % for calculating minipage widths
% Correct order of tables after \paragraph or \subparagraph
\usepackage{etoolbox}
\makeatletter
\patchcmd\longtable{\par}{\if@noskipsec\mbox{}\fi\par}{}{}
\makeatother
% Allow footnotes in longtable head/foot
\IfFileExists{footnotehyper.sty}{\usepackage{footnotehyper}}{\usepackage{footnote}}
\makesavenoteenv{longtable}
\usepackage{graphicx}
\makeatletter
\def\maxwidth{\ifdim\Gin@nat@width>\linewidth\linewidth\else\Gin@nat@width\fi}
\def\maxheight{\ifdim\Gin@nat@height>\textheight\textheight\else\Gin@nat@height\fi}
\makeatother
% Scale images if necessary, so that they will not overflow the page
% margins by default, and it is still possible to overwrite the defaults
% using explicit options in \includegraphics[width, height, ...]{}
\setkeys{Gin}{width=\maxwidth,height=\maxheight,keepaspectratio}
% Set default figure placement to htbp
\makeatletter
\def\fps@figure{htbp}
\makeatother

% NOTE: KOMAscript settings for header and footer
\usepackage[headsepline = 1.5pt, footsepline = 1.5pt]{scrlayer-scrpage}
\setkomafont{pagenumber}{\bfseries}
\setkomafont{pagehead}{\bfseries}
\setkomafont{pagefoot}{\itshape}
\lohead*{Reading Evidence Instructions}
\rohead*{\pagemark}
\cofoot*{"Scientists have become the bearers of the torch of discovery in our quest for knowledge" - Stephen Hawking}

% NOTE: Color settings
\usepackage{color}
\definecolor{gvblue}{HTML}{0032A0} % NOTE: Setting a new color
\renewcommand\thesection{\color{gvblue}\arabic{section}} % NOTE: Change section number color

% NOTE: Symbols, Fonts, and typesetting
\usepackage{bbding} % NOTE: For checkmark with \Checkmark
\usepackage{mathtools, amsthm, amssymb, amsfonts}
\usepackage{lmodern}
\usepackage[utf8]{inputenc}

% NOTE: Double-spacing and indents
% \usepackage{setspace} % NOTE: Allows for specifying spaces
% \usepackage{indentfirst} % NOTE: Indent first paragraph

\usepackage{multicol}
\makeatletter
\@ifpackageloaded{caption}{}{\usepackage{caption}}
\AtBeginDocument{%
\ifdefined\contentsname
  \renewcommand*\contentsname{Table of contents}
\else
  \newcommand\contentsname{Table of contents}
\fi
\ifdefined\listfigurename
  \renewcommand*\listfigurename{List of Figures}
\else
  \newcommand\listfigurename{List of Figures}
\fi
\ifdefined\listtablename
  \renewcommand*\listtablename{List of Tables}
\else
  \newcommand\listtablename{List of Tables}
\fi
\ifdefined\figurename
  \renewcommand*\figurename{Figure}
\else
  \newcommand\figurename{Figure}
\fi
\ifdefined\tablename
  \renewcommand*\tablename{Table}
\else
  \newcommand\tablename{Table}
\fi
}
\@ifpackageloaded{float}{}{\usepackage{float}}
\floatstyle{ruled}
\@ifundefined{c@chapter}{\newfloat{codelisting}{h}{lop}}{\newfloat{codelisting}{h}{lop}[chapter]}
\floatname{codelisting}{Listing}
\newcommand*\listoflistings{\listof{codelisting}{List of Listings}}
\makeatother
\makeatletter
\makeatother
\makeatletter
\@ifpackageloaded{caption}{}{\usepackage{caption}}
\@ifpackageloaded{subcaption}{}{\usepackage{subcaption}}
\makeatother

\usepackage{hyphenat}
\usepackage{ifthen}
\usepackage{calc}
\usepackage{calculator}



\usepackage{graphicx}
\usepackage{geometry}
\usepackage{afterpage}
\usepackage{tikz}
\usetikzlibrary{calc}
\usetikzlibrary{fadings}
\usepackage[pagecolor=none]{pagecolor}


% Set the titlepage font families







% Set the coverpage font families


\ifLuaTeX
\usepackage[bidi=basic]{babel}
\else
\usepackage[bidi=default]{babel}
\fi
\babelprovide[main,import]{american}
% get rid of language-specific shorthands (see #6817):
\let\LanguageShortHands\languageshorthands
\def\languageshorthands#1{}
\ifLuaTeX
  \usepackage{selnolig}  % disable illegal ligatures
\fi
\usepackage{bookmark}

\IfFileExists{xurl.sty}{\usepackage{xurl}}{} % add URL line breaks if available
\urlstyle{same} % disable monospaced font for URLs
\hypersetup{
  pdftitle={PSY-300 Reading Evidence Instructions},
  pdfauthor={Quinton Quagliano, M.S.},
  pdflang={en-US},
  pdfsubject={PSY-300 Reading Evidence Instructions},
  pdfkeywords={syllabus},
  colorlinks=true,
  linkcolor={gvblue},
  filecolor={Maroon},
  citecolor={gvblue},
  urlcolor={gvblue},
  pdfcreator={LaTeX via pandoc}}


\title{Reading Evidence Instructions - Fall 2024}
\usepackage{etoolbox}
\makeatletter
\providecommand{\subtitle}[1]{% add subtitle to \maketitle
  \apptocmd{\@title}{\par {\large #1 \par}}{}{}
}
\makeatother
\subtitle{PSY-300: Research Methods in Psychology}
\author{Quinton Quagliano, M.S.}
\date{Friday, September 27, 2024}

\begin{document}
%%%%% begin titlepage extension code


\begin{titlepage}

%%% TITLE PAGE START

% Set up alignment commands
%Page
\newcommand{\titlepagepagealign}{
\ifthenelse{\equal{center}{right}}{\raggedleft}{}
\ifthenelse{\equal{center}{center}}{\centering}{}
\ifthenelse{\equal{center}{left}}{\raggedright}{}
}


\newcommand{\titleandsubtitle}{
% Title and subtitle
{{\huge{\bfseries{\nohyphens{Reading Evidence Instructions - Fall
2024}}}}\par
}%

\vspace{\betweentitlesubtitle}
{
{\Large{\nohyphens{PSY-300: Research Methods in Psychology}}}\par
}}
\newcommand{\titlepagetitleblock}{
\newcommand{\HRule}{\rule{\linewidth}{0.5mm}} 

\HRule\\[0.4cm]

\titleandsubtitle

\HRule\\
}
\newcommand{\authorstyle}[1]{{\small{#1}}}

\newcommand{\affiliationstyle}[1]{{\small{#1}}}

\newcommand{\titlepageauthorblock}{
\newlength{\miniA}
\setlength{\miniA}{0pt}
\newlength{\namelen}
\settowidth{\namelen}{Quinton Quagliano,
M.S.}\setlength{\miniA}{\maxof{\miniA}{\namelen}}
\setlength{\miniA}{\miniA+0.05\textwidth}
\newlength{\miniB}
\setlength{\miniB}{0.99\textwidth - \miniA}
\begin{minipage}{\miniA}
\begin{flushleft}
{\authorstyle{Quinton Quagliano, M.S.}}
\end{flushleft}
\end{minipage}
\begin{minipage}{\miniB}
\begin{flushright}
{\affiliationstyle{Department of Psychology
\\}}
\end{flushright}
\end{minipage}}

\newcommand{\titlepageaffiliationblock}{}
\newcommand{\headerstyled}{%
{\textsc{\LARGE{}}}
}
\newcommand{\footerstyled}{%
{}
}
\newcommand{\datestyled}{%
{\large{Friday, September 27, 2024}}
}


\newcommand{\titlepageheaderblock}{\headerstyled}

\newcommand{\titlepagefooterblock}{
\footerstyled
}

\newcommand{\titlepagedateblock}{
\datestyled
}

%set up blocks so user can specify order
\newcommand{\titleblock}{\newlength{\betweentitlesubtitle}
\setlength{\betweentitlesubtitle}{\baselineskip}
{

{\titlepagetitleblock}
}

\vspace{1.5cm}
}

\newcommand{\authorblock}{{\titlepageauthorblock}

\vspace{1.5cm}
}

\newcommand{\affiliationblock}{{\titlepageaffiliationblock}

\vspace{0pt}
}

\newcommand{\logoblock}{{\includegraphics[width=250pt]{gvlogo.png}}

\vspace{2\baselineskip}
}

\newcommand{\footerblock}{}

\newcommand{\dateblock}{{\titlepagedateblock}

\vspace{0pt}
}

\newcommand{\headerblock}{}

\thispagestyle{empty} % no page numbers on titlepages


\newlength{\minipagewidth}
\setlength{\minipagewidth}{\textwidth}
\raggedright % single minipage
% [position of box][box height][inner position]{width}
% [s] means stretch out vertically; assuming there is a vfill
\begin{minipage}[b][\textheight][s]{\minipagewidth}
\titlepagepagealign
\headerblock

\logoblock

\titleblock

\authorblock
\par

\end{minipage}\ifthenelse{\equal{}{right} \OR \equal{}{leftright} }{
\hspace{\B}
\vrulecode}{}
\clearpage
%%% TITLE PAGE END
\end{titlepage}
\setcounter{page}{1}

%%%%% end titlepage extension code

% NOTE: Better hold for LaTeX tables
\floatplacement{table}{H}

% NOTE: Set KOMAscript headings
\pagestyle{scrheadings}

% NOTE: Double-spacing and indents
% \doublespacing
\frenchspacing % NOTE: Only single space after period
% \setlength{\parindent}{20pt} % NOTE: Amount of spacing in indent

\renewcommand*\contentsname{Table of Contents}
{
\hypersetup{linkcolor=}
\setcounter{tocdepth}{2}
\tableofcontents
}

\newpage{}

\section{Assignment Instructions}\label{assignment-instructions}

``Reading Evidence'' for this course, is some record of your note-taking
or engagement with the \textbf{assigned textbook readings} for each
week. I recognize that notes can look very different from person to
person, and I will respect a variety of approaches and still award full
points. However, it is most critical to me that you demonstrate a
\textbf{complete and thorough reading of the chapter}, meaning I do
expect you to at least reference content from each and every section and
subsection in the chapter. Your notes do not have to include information
from sidebars or example questions, e.g., ``test your knowledge'' - but
you are welcome to include these if you think it will be helpful to you.
You do not have to be overly deep and detailed, and it is okay to
admit/show confusion on a certain topic - but please do give a good
effort. I would encourage you to also connect the text to activities
that we do in class and also share your own thoughts about concepts,
rather than just writing things from the book.

When attempting to get full points, you should ask yourself, ``Do these
notes seem like someone who has read the textbook, \emph{not} just
attended the lecture?'' If your notes appear to just a replication of
content from class slides or purely based on the lecture (i.e., not
reading), your work will receive 0 points. You should not be merely
reading the textbook and watching the lectures, but actively engaging
with them by writing and recording information.

These notes can be in a digital (typed) form, or a physical form
(handwritten). If you take notes by hand I will have you submit photos
(.jpg or .png) of your notes, and if you type your notes, please provide
them in a .docx or .pdf format.

I require that you submit reading evidence for any 5 of the 14 chapters
from this class. While I encourage you to take good notes on all the
chapters (to help in quizzes and exam review), I will only grade the 5
you submit. Each submission will be worth 10 points, for a total of 50
points in this category. I will grade these leniently so long as you
have covered all the content (as I described above), and provide
feedback if there are any ways I think you could make more use of the
notes. You do not \emph{have} to take my advice that I offer on your
submission, I just offer it for your consideration. If your work is
below standard for full-points, I will make that clear to you and inform
you of what changes are needed for full points.

As for due dates, I require that you submit at least 1 of the reading
evidences for 1 of the chapters by Sept 10, so that I can give early
feedback to get everyone started. Then, the remainder of the evidences
are due near the end of the semester, on Nov 26.

\section{Tips for Better Notes}\label{tips-for-better-notes}

There are many good ways to take notes on text and lecture material, and
I cannot decide what is most suitable for you. I will provide some
initial tips to help you get started, and, hopefully, give you some
ideas. I will also provide an example of my own notes on a chapter from
a different textbook, so you may use that as a source of inspiration if
you choose.

\subsection{Keep Consistent Style}\label{keep-consistent-style}

Throughout each unit, reading, lecture, and week, I would encourage you
to keep your style of note-taking relatively consistent and organized.
This will be especially helpful during open-notes quizzes and when you
are filling out exam study guides. If you choose to use bolding,
underlines, italics, and/or colors, please do so in an informative and
intuitive way.

\subsection{Use Structure}\label{use-structure}

I use indented bullet points like this:

\begin{itemize}
\tightlist
\item
  Section 1

  \begin{itemize}
  \tightlist
  \item
    Subsection 1
  \item
    Subsection 2

    \begin{itemize}
    \tightlist
    \item
      Notes
    \end{itemize}
  \end{itemize}
\end{itemize}

to keep topics clearly organized, like you would find in a presentation
or chapter reading. While you do not have to follow that style exactly,
do think about using some structure to keep related concepts grouped
together. Along with the styling options I highlighted above, you have a
nice variety of ways to keep your notes in a readable, accessible
manner.

\subsection{Think Abstract}\label{think-abstract}

It is very easy to just take notes on what is written exactly in the
book and nothing more (in a very concrete manner). It is advantageous to
try to connect concepts to examples and add your own thoughts in. I
personally like giving ``commentary'' on what I read as I write, so I
basically have to take the content in and react to it somehow. Doing
things like this can aid in memorization and also help connect topics
across units. You learn best when you find a way to make the content
\emph{applicable to you}.

\subsection{Read BEFORE Lectures}\label{read-before-lectures}

While this is a hard habit to make, it is one likely to pay dividends. I
would strongly suggest that you read the textbook chapter for a week
before you come to the lecture with the corresponding content. It will
help familiarize you with the concepts and prompt you to consider what
questions you may have. That way, in class, you can readily ask me about
things that are confusing. I like to take notes from the book, and then
``mark them up'' with content from the lecture to help connect the
spoken content with the written material.




\end{document}
