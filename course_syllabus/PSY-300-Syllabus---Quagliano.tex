% Options for packages loaded elsewhere
\PassOptionsToPackage{unicode}{hyperref}
\PassOptionsToPackage{hyphens}{url}
\PassOptionsToPackage{dvipsnames,svgnames,x11names}{xcolor}
%
\documentclass[
  12pt,
  letterpaper,
]{scrartcl}

\usepackage{amsmath,amssymb}
\usepackage{iftex}
\ifPDFTeX
  \usepackage[T1]{fontenc}
  \usepackage[utf8]{inputenc}
  \usepackage{textcomp} % provide euro and other symbols
\else % if luatex or xetex
  \usepackage{unicode-math}
  \defaultfontfeatures{Scale=MatchLowercase}
  \defaultfontfeatures[\rmfamily]{Ligatures=TeX,Scale=1}
\fi
\usepackage{lmodern}
\ifPDFTeX\else  
    % xetex/luatex font selection
\fi
% Use upquote if available, for straight quotes in verbatim environments
\IfFileExists{upquote.sty}{\usepackage{upquote}}{}
\IfFileExists{microtype.sty}{% use microtype if available
  \usepackage[protrusion = true]{microtype}
  \UseMicrotypeSet[protrusion]{basicmath} % disable protrusion for tt fonts
}{}
\makeatletter
\@ifundefined{KOMAClassName}{% if non-KOMA class
  \IfFileExists{parskip.sty}{%
    \usepackage{parskip}
  }{% else
    \setlength{\parindent}{0pt}
    \setlength{\parskip}{6pt plus 2pt minus 1pt}}
}{% if KOMA class
  \KOMAoptions{parskip=half}}
\makeatother
\usepackage{xcolor}
\usepackage[margin = 1in]{geometry}
\setlength{\emergencystretch}{3em} % prevent overfull lines
\setcounter{secnumdepth}{2}
% Make \paragraph and \subparagraph free-standing
\makeatletter
\ifx\paragraph\undefined\else
  \let\oldparagraph\paragraph
  \renewcommand{\paragraph}{
    \@ifstar
      \xxxParagraphStar
      \xxxParagraphNoStar
  }
  \newcommand{\xxxParagraphStar}[1]{\oldparagraph*{#1}\mbox{}}
  \newcommand{\xxxParagraphNoStar}[1]{\oldparagraph{#1}\mbox{}}
\fi
\ifx\subparagraph\undefined\else
  \let\oldsubparagraph\subparagraph
  \renewcommand{\subparagraph}{
    \@ifstar
      \xxxSubParagraphStar
      \xxxSubParagraphNoStar
  }
  \newcommand{\xxxSubParagraphStar}[1]{\oldsubparagraph*{#1}\mbox{}}
  \newcommand{\xxxSubParagraphNoStar}[1]{\oldsubparagraph{#1}\mbox{}}
\fi
\makeatother

\providecommand{\tightlist}{%
  \setlength{\itemsep}{0pt}\setlength{\parskip}{0pt}}\usepackage{longtable,booktabs,array}
\usepackage{calc} % for calculating minipage widths
% Correct order of tables after \paragraph or \subparagraph
\usepackage{etoolbox}
\makeatletter
\patchcmd\longtable{\par}{\if@noskipsec\mbox{}\fi\par}{}{}
\makeatother
% Allow footnotes in longtable head/foot
\IfFileExists{footnotehyper.sty}{\usepackage{footnotehyper}}{\usepackage{footnote}}
\makesavenoteenv{longtable}
\usepackage{graphicx}
\makeatletter
\def\maxwidth{\ifdim\Gin@nat@width>\linewidth\linewidth\else\Gin@nat@width\fi}
\def\maxheight{\ifdim\Gin@nat@height>\textheight\textheight\else\Gin@nat@height\fi}
\makeatother
% Scale images if necessary, so that they will not overflow the page
% margins by default, and it is still possible to overwrite the defaults
% using explicit options in \includegraphics[width, height, ...]{}
\setkeys{Gin}{width=\maxwidth,height=\maxheight,keepaspectratio}
% Set default figure placement to htbp
\makeatletter
\def\fps@figure{htbp}
\makeatother

% NOTE: KOMAscript settings for header and footer
\usepackage[headsepline = 1.5pt, footsepline = 1.5pt]{scrlayer-scrpage}
\setkomafont{pagenumber}{\bfseries}
\setkomafont{pagehead}{\bfseries}
\setkomafont{pagefoot}{\itshape}
\lohead*{PSY-300 Syllabus}
\rohead*{\pagemark}
\cofoot*{"I don't mind not knowing. It doesn't scare me." - Richard Feynman }

% NOTE: Color settings
\usepackage{color}
\definecolor{gvblue}{HTML}{0032A0} % NOTE: Setting a new color
\renewcommand\thesection{\color{gvblue}\arabic{section}} % NOTE: Change section number color

% NOTE: Symbols, Fonts, and typesetting
\usepackage{bbding} % NOTE: For checkmark with \Checkmark
\usepackage{mathtools, amsthm, amssymb, amsfonts}
\usepackage{lmodern}
\usepackage[utf8]{inputenc}

% NOTE: Double-spacing and indents
% \usepackage{setspace} % NOTE: Allows for specifying spaces
% \usepackage{indentfirst} % NOTE: Indent first paragraph

\usepackage{multicol}
\makeatletter
\@ifpackageloaded{caption}{}{\usepackage{caption}}
\AtBeginDocument{%
\ifdefined\contentsname
  \renewcommand*\contentsname{Table of contents}
\else
  \newcommand\contentsname{Table of contents}
\fi
\ifdefined\listfigurename
  \renewcommand*\listfigurename{List of Figures}
\else
  \newcommand\listfigurename{List of Figures}
\fi
\ifdefined\listtablename
  \renewcommand*\listtablename{List of Tables}
\else
  \newcommand\listtablename{List of Tables}
\fi
\ifdefined\figurename
  \renewcommand*\figurename{Figure}
\else
  \newcommand\figurename{Figure}
\fi
\ifdefined\tablename
  \renewcommand*\tablename{Table}
\else
  \newcommand\tablename{Table}
\fi
}
\@ifpackageloaded{float}{}{\usepackage{float}}
\floatstyle{ruled}
\@ifundefined{c@chapter}{\newfloat{codelisting}{h}{lop}}{\newfloat{codelisting}{h}{lop}[chapter]}
\floatname{codelisting}{Listing}
\newcommand*\listoflistings{\listof{codelisting}{List of Listings}}
\makeatother
\makeatletter
\makeatother
\makeatletter
\@ifpackageloaded{caption}{}{\usepackage{caption}}
\@ifpackageloaded{subcaption}{}{\usepackage{subcaption}}
\makeatother

\usepackage{hyphenat}
\usepackage{ifthen}
\usepackage{calc}
\usepackage{calculator}



\usepackage{graphicx}
\usepackage{geometry}
\usepackage{afterpage}
\usepackage{tikz}
\usetikzlibrary{calc}
\usetikzlibrary{fadings}
\usepackage[pagecolor=none]{pagecolor}


% Set the titlepage font families







% Set the coverpage font families


\ifLuaTeX
\usepackage[bidi=basic]{babel}
\else
\usepackage[bidi=default]{babel}
\fi
\babelprovide[main,import]{american}
% get rid of language-specific shorthands (see #6817):
\let\LanguageShortHands\languageshorthands
\def\languageshorthands#1{}
\ifLuaTeX
  \usepackage{selnolig}  % disable illegal ligatures
\fi
\usepackage{bookmark}

\IfFileExists{xurl.sty}{\usepackage{xurl}}{} % add URL line breaks if available
\urlstyle{same} % disable monospaced font for URLs
\hypersetup{
  pdftitle={PSY-300 Syllabus},
  pdfauthor={Quinton Quagliano, M.S.},
  pdflang={en-US},
  pdfsubject={PSY-300 Syllabus},
  pdfkeywords={syllabus},
  colorlinks=true,
  linkcolor={gvblue},
  filecolor={Maroon},
  citecolor={gvblue},
  urlcolor={gvblue},
  pdfcreator={LaTeX via pandoc}}


\title{PSY-300: Research Methods in Psychology}
\usepackage{etoolbox}
\makeatletter
\providecommand{\subtitle}[1]{% add subtitle to \maketitle
  \apptocmd{\@title}{\par {\large #1 \par}}{}{}
}
\makeatother
\subtitle{Syllabus Fall 2024}
\author{Quinton Quagliano, M.S.}
\date{Sunday, August 25, 2024}

\begin{document}
%%%%% begin titlepage extension code


\begin{titlepage}

%%% TITLE PAGE START

% Set up alignment commands
%Page
\newcommand{\titlepagepagealign}{
\ifthenelse{\equal{center}{right}}{\raggedleft}{}
\ifthenelse{\equal{center}{center}}{\centering}{}
\ifthenelse{\equal{center}{left}}{\raggedright}{}
}


\newcommand{\titleandsubtitle}{
% Title and subtitle
{{\huge{\bfseries{\nohyphens{PSY-300: Research Methods in
Psychology}}}}\par
}%

\vspace{\betweentitlesubtitle}
{
{\Large{\nohyphens{Syllabus Fall 2024}}}\par
}}
\newcommand{\titlepagetitleblock}{
\newcommand{\HRule}{\rule{\linewidth}{0.5mm}} 

\HRule\\[0.4cm]

\titleandsubtitle

\HRule\\
}
\newcommand{\authorstyle}[1]{{\small{#1}}}

\newcommand{\affiliationstyle}[1]{{\small{#1}}}

\newcommand{\titlepageauthorblock}{
\newlength{\miniA}
\setlength{\miniA}{0pt}
\newlength{\namelen}
\settowidth{\namelen}{Quinton Quagliano,
M.S.}\setlength{\miniA}{\maxof{\miniA}{\namelen}}
\setlength{\miniA}{\miniA+0.05\textwidth}
\newlength{\miniB}
\setlength{\miniB}{0.99\textwidth - \miniA}
\begin{minipage}{\miniA}
\begin{flushleft}
{\authorstyle{Quinton Quagliano, M.S.}}
\end{flushleft}
\end{minipage}
\begin{minipage}{\miniB}
\begin{flushright}
{\affiliationstyle{Department of Psychology
\\}}
\end{flushright}
\end{minipage}}

\newcommand{\titlepageaffiliationblock}{}
\newcommand{\headerstyled}{%
{\textsc{\LARGE{}}}
}
\newcommand{\footerstyled}{%
{}
}
\newcommand{\datestyled}{%
{\large{Sunday, August 25, 2024}}
}


\newcommand{\titlepageheaderblock}{\headerstyled}

\newcommand{\titlepagefooterblock}{
\footerstyled
}

\newcommand{\titlepagedateblock}{
\datestyled
}

%set up blocks so user can specify order
\newcommand{\titleblock}{\newlength{\betweentitlesubtitle}
\setlength{\betweentitlesubtitle}{\baselineskip}
{

{\titlepagetitleblock}
}

\vspace{1.5cm}
}

\newcommand{\authorblock}{{\titlepageauthorblock}

\vspace{1.5cm}
}

\newcommand{\affiliationblock}{{\titlepageaffiliationblock}

\vspace{0pt}
}

\newcommand{\logoblock}{{\includegraphics[width=250pt]{gvlogo.png}}

\vspace{2\baselineskip}
}

\newcommand{\footerblock}{}

\newcommand{\dateblock}{{\titlepagedateblock}

\vspace{0pt}
}

\newcommand{\headerblock}{}

\thispagestyle{empty} % no page numbers on titlepages


\newlength{\minipagewidth}
\setlength{\minipagewidth}{\textwidth}
\raggedright % single minipage
% [position of box][box height][inner position]{width}
% [s] means stretch out vertically; assuming there is a vfill
\begin{minipage}[b][\textheight][s]{\minipagewidth}
\titlepagepagealign
\headerblock

\logoblock

\titleblock

\authorblock
\par

\end{minipage}\ifthenelse{\equal{}{right} \OR \equal{}{leftright} }{
\hspace{\B}
\vrulecode}{}
\clearpage
%%% TITLE PAGE END
\end{titlepage}
\setcounter{page}{1}

%%%%% end titlepage extension code

% NOTE: Better hold for LaTeX tables
\floatplacement{table}{H}

% NOTE: Set KOMAscript headings
\pagestyle{scrheadings}

% NOTE: Double-spacing and indents
% \doublespacing
\frenchspacing % NOTE: Only single space after period
% \setlength{\parindent}{20pt} % NOTE: Amount of spacing in indent

\renewcommand*\contentsname{Table of Contents}
{
\hypersetup{linkcolor=}
\setcounter{tocdepth}{2}
\tableofcontents
}

\newpage{}

\section{Contact Information}\label{contact-information}

\textbf{Instructor:} Quinton Quagliano, M.S. (he/him/his), Adjunct
Professor of Psychology

\textbf{Office:} AuSable Hall (ASH) 1307 - Shared Adjunct Faculty Office

\textbf{Office Hours:} Fridays, 2:00pm - 5:00pm EST

\textbf{Phone:} 616-331-2976 (Zoom Phone; Not Recommended)

\textbf{Email:} \href{mailto:quagliaq@gvsu.edu}{QuagliaQ@gvsu.edu}
(Strongly Recommended)

\section{Course Overview}\label{course-overview}

\subsection{Prerequisites}\label{prerequisites}

\begin{itemize}
\tightlist
\item
  \href{https://www.gvsu.edu/catalog/course/wrt-150.htm}{WRT-150}
\item
  \href{https://www.gvsu.edu/catalog/course/psy-101.htm}{PSY-101}
\item
  \href{https://www.gvsu.edu/catalog/course/sta-215.htm}{STA-215} or
  \href{https://www.gvsu.edu/catalog/course/sta-312.htm}{STA-312}
\item
  From these courses students should have a reasonable familiarity with
  fundamental concepts in psychology and statistics prior to this course
\item
  Students should also be prepared to write in a coherent and clear way,
  and be able to reasonably integrate evidence into arguments and
  statements
\end{itemize}

\subsection{Successive Courses}\label{successive-courses}

\begin{itemize}
\tightlist
\item
  \href{https://www.gvsu.edu/catalog/course/psy-350.htm}{PSY-350}
\item
  \href{https://www.gvsu.edu/catalog/course/psy-400.htm}{PSY-400}
\item
  These courses will expand on the students' skills established in this
  course, in more applied projects and assignments
\end{itemize}

\subsection{Meeting Information}\label{meeting-information}

\textbf{Modality:} In-person, synchronous, traditional classroom setting

\textbf{Location:} AuSable Hall (ASH) 1142 Classroom

\textbf{Time:} Tuesdays, 6:00pm - 8:50pm EST

\subsection{Textbooks}\label{textbooks}

\subsubsection{Required Text}\label{required-text}

Morling, B. (2021). \emph{Research methods in psychology: Evaluating a
world of information} (Fourth edition). W. W. Norton \& Company.

This text is provided through the GVSU SAVE program, learn more about
how it works here: \url{https://lakerstore.gvsu.edu/GVSUSAVEStudents}.
Note that you may have to opt out of this program, by Sept.~6th, if you
wish to purchase a physical copy. The link to access this text from
Blackboard will be near the top of the page.

\subsubsection{Recommended Text
(Optional)}\label{recommended-text-optional}

American Psychological Association (2020). \emph{Publication manual of
the {American Psychological Association}: The official guide to {APA}
style} (Seventh edition). American Psychological Association.
\url{https://doi.org/10.1037/0000165-000}

I would recommend you purchase a copy of this text, for reference during
this course and future projects. I still regularly use my own during
professional work, and it is a wealth of resources and suggestions for
better scientific writing.

\subsection{Course Description}\label{course-description}

3-credit course.

This course is focused on developing an understanding of philosophy,
logic, and procedures of good science. We will cover a variety of
experimental and observational methodologies in psychology, and discuss
their respective strengths and weaknesses. Attention will be paid to
developing rigorous procedures in psychological research and
assessing/critiquing scientific literature. We will also explore ethics,
bias, and validity in study design. Scientific writing and critical
reading will be essential skills in this course.

\subsection{Learning Objectives}\label{learning-objectives}

At the end of this course students are expected to be able to:

\subsubsection{Course Objectives}\label{course-objectives}

\begin{itemize}
\tightlist
\item
  \textbf{Comprehension/Describe:} Describe the details of summarize the
  essence of research articles
\item
  \textbf{Application/Write:} Write clearly and cogently in a scientific
  way
\item
  \textbf{Evaluation/Evaluate:} Evaluate strengths and weaknesses of
  empirical studies consistent with the standards of psychological
  science
\end{itemize}

\subsubsection{Professor's Objectives}\label{professors-objectives}

\begin{itemize}
\tightlist
\item
  Evaluate, summarize, and maturely discuss the merits and limitations
  of research studies
\item
  Conduct effective literature reviews that are inclusive of all
  relevant evidence
\item
  Make testable hypotheses supported by appropriate study design and
  references
\item
  Write scientific papers clearly in the APA 7th edition style
\item
  Understand the importance of a high-standard of ethics in research
\end{itemize}

\subsection{Course Format}\label{course-format}

This is an in-person, synchronous course that will meet in a traditional
classroom setting for roughly 3 hours every Tuesday of the Fall
semester. Class periods will consist of a combination of activities that
support the engagement, attention, and learning of students for the
entire 3 hours. Planned activities include: lectures, in-class
discussion, demonstrations, \hyperref[quizzes]{Quizzes},
\hyperref[exams]{Exams}, etc. Breaks will be provided intermittently
throughout the course periods. Prepare for class periods to last the
entirety of the scheduled course time. In-person classes will not be
recorded, but some materials will be available to students who may have
missed a class period due to excused absences (see {[}Attendence{]} and
\hyperref[due-dates]{Due Dates}).

Outside of class, students should work diligently and efficiently on
assigned coursework. \textbf{Textbook readings, as indicated on the
\hyperref[schedule]{Schedule}, are always required and not optional.}
The readings will be tested upon and are essential for being prepared
for content during the class period. I also require you submit your
textbook notes for a subset of the chapters (i.e.,
``\hyperref[reading-evidence]{Reading Evidence}'') In this course, my
lectures and the assigned readings will be closely linked and should be
viewed as complimentary to each other. I will ensure lectures cover the
key content from each chapter. We only meet once every week, so students
are expected to manage their own time well, and to finish all
assignments in advance of the deadlines provided in the
\hyperref[schedule]{Schedule}.

\section{Course Policies}\label{course-policies}

This course is subject to the general GVSU policies listed at
\url{http://www.gvsu.edu/coursepolicies/}. Please review all content on
that webpage in addition to what is written here. If you have questions
or concern about any of these policies, please contact me.

\subsection{Contacting Me}\label{contacting-me}

Prior to contacting me, please review this syllabus, relevant assignment
sheets, and presentations and make sure the answer you seek is not
already there first. While I always invite questions and communication,
I ask that you are discerning in using your available resources first.

If you do need to reach out, refer to my
\hyperref[contact-information]{Contact Information}. \textbf{Please
prioritize using email, and include your full name, course number, and
section number in the subject line, as well as a description of what you
need help with}. An example would be:

``Re: Riley Quagliano, PSY-300 Section 10, SPSS Help for Assignment 2''

Be thorough in your email and tell me what resources you have already
used to try to address your concerns. Detail is helpful so that I am
able to fully understand and attend to your concern. I am generally able
to send responses within 24 hours during weekdays, but please allow me
up to 48 hours to respond. If I haven't responded within 48 hours
(during weekdays), please send me a remainder email - I promise I am not
ignoring you! If you email me during the weekend or later on a Friday, I
cannot guarantee a response any earlier than Monday.

Of course, you are always welcome to ask to schedule a video chat or
office visit, or just stop by during my listed office hours - I'm happy
to receive visitors for questions about the course or to chat about
other things. I usually hang around campus after the scheduled course
time, so you may catch me after class for shorter conversations.

\subsection{Integrity, Plagiarism, \& Academic
Dishonesty}\label{integrity-plagiarism-academic-dishonesty}

This course will challenge students to grow skills in responsible and
ethical research, in which one of the most essential abilities is
writing in a way that gives proper attribution and credit. When drafting
a paper, presentation, report, or any type of assignment we
\textbf{must} take care to use our own words and thoughts, and cite the
scholars that we build from. In the academic world, plagiarism,
fabrication, and confabulation are some of the most egregious crimes,
and have resulted in de-funding of labs, loss of grants, and the
destruction of professional reputations.

I will hold students to the same high standards of the professional
research world, as I wish to ensure all of you are able to leave this
course confident in your ability to conduct and write ethical research.
I will help all students steadily build these skills, and do expect
early mistakes that need correction. However, any evidence of
intentional or negligent plagiarism or academic dishonesty will be
handled in line with the GVSU Student Code.

As described by the GVSU Student Code, ``Offering the work of someone
else as one's own is plagiarism\ldots{}'' ``Any ideas or material taken
from another source for either written or oral presentation must be
fully acknowledged.'' ``Depending on the instructor's judgment of the
particular case, he/she may \ldots{} give a failing grade for the
\ldots{} entire course.'' Simply rearranging the words or substituting
synonyms in the original source is still plagiarism. Details about the
APA method for citing research will be provided during the course.
Furthermore, students should not self-plagiarize, that is, reuse their
own work from another course.

Students are permitted to be in study groups and learn from one another,
but all submitted work should be distinct and unique to each individual.
I would recommend you only study in groups for quizzes or exams, but
complete other work (e.g., Article Critique, Research Proposal, etc.) as
an individual, to avoid unintentional plagiarism. Students in other
sections of this course, with other professors, may not have perfectly
aligned content with the pace of this section.

\subsection{Use of AI Tools}\label{use-of-ai-tools}

The following statement should be understood as constituting my policy
on use of AI tools in this course:

AI Policy Statement by David A. Joyner
(\href{mailto:davidjoyner@fediscience.org}{\nolinkurl{davidjoyner@fediscience.org}})

\begin{quote}
``We treat AI-based assistance, such as ChatGPT and Copilot, the same
way we treat collaboration with other people: you are welcome to talk
about your ideas and work with other people, both inside and outside the
class, as well as with AI-based assistants.

However, all work you submit must be your own. You should never include
in your assignment anything that was not written directly by you without
proper citation (including quotation marks and in-line citation for
direct quotes).

Including anything you did not write in your assignment without proper
citation will be treated as an academic misconduct case. If you are
unsure where the line is between collaborating with AI and copying from
AI, we recommend the following:

\begin{enumerate}
\def\labelenumi{\arabic{enumi}.}
\item
  Never hit ``Copy'' within your conversation with an AI assistant. You
  can copy your own work into your conversation, but do not copy
  anything from the conversation back into your assignment. Instead, use
  your interaction with the AI assistant as a learning experience, then
  let your assignment reflect your improved understanding.
\item
  Do not have your assignment and the AI agent open at the same time.
  Similar to above, use your conversation with the AI as a learning
  experience, then close the interaction down, open your assignment, and
  let your assignment reflect your revised knowledge. This includes
  avoiding using AI directly integrated into your composition
  environment: just as you should not let a classmate write content
  directly into your submission, so also avoid using tools that directly
  add content to your submission.''
\end{enumerate}
\end{quote}

While ``AI'', as we know it, may be a useful tool for learning and
troubleshooting, it is never an acceptable replacement for graded and
professional work. For the work in this course, I'd strongly recommend
you rely upon class content, the textbook, and peer-reviewed scientific
research to study and build your skills. If I have reason to believe
that AI tools have been used to generate work passed off as your own, I
will investigate it the same as any other form of academic dishonesty
(see \hyperref[integrity-plagiarism-academic-dishonesty]{Integrity,
Plagiarism, \& Academic Dishonesty}). I implore you to maintain a high
level of integrity in your work and I will take appropriate measures to
detect and investigate unethical use of AI tools.

\subsection{Respectful and Inclusive
Environment}\label{respectful-and-inclusive-environment}

All students, in all their virtual and physical interactions with myself
and one another, are expected to treat each other in a mindful and
professional manner. Please be respectful of your classmates' diverse
identities, backgrounds, and beliefs. Even in disagreements or tense
discussion, students should remain constructive in their arguments, and
not personally attack one another. If at any point you feel that another
student has acted maliciously, aggressively, or disrespectful towards
you, please notify me immediately, so I may address it. Additionally, if
you have concern about my own conduct, please do let me know so, I can
correct it. Those that continue to act in a way that is harmful to the
classroom environment may be dismissed from the course and referred to
college administration for further discipline.

\subsection{Electronic Devices}\label{electronic-devices}

Students are permitted to use computers or other devices to take notes
and complete assignments during the class period. In fact, I encourage
you to bring internet-capable devices to our class meetings, as they
will likely be helpful in certain in-class activities and quizzes.
However, refrain from doing any activities during class that distract
those around you (e.g., watching videos, playing video games, listening
to music, distracting loud keyboards, etc.) If I suspect that you are
harming the learning experience of other students, I will ask you to
either cease the behavior or switch to pen-and-paper for note-taking.
Please ensure all electronic devices are silenced during class (medical
devices that must alert for the well-being of the user are exempt).
Regardless of whether you intend on using an electronic device for
class, I'd recommend having a pen/pencil and paper, in case a device
crashes or is out of battery.

\subsection{Food and Drink}\label{food-and-drink}

It is fine to eat and drink during the class period (except for during
exams and quizzes), as long as you are able to do so quietly and without
interrupting others. Please do not make a mess, as to be respectful to
those who diligently clean our rooms.

\subsection{Course Materials and
Recordings}\label{course-materials-and-recordings}

For any materials created by me, such as notes, slides, handouts,
assignments, etc. - do not share these materials with students who are
not in the section, now or in the future. You may keep them for your own
personal reference. I cannot allow students to keep copies of exams or
quizzes, even after they are graded. Resources that I provide that are
already public or semi-public, like articles and webpages, are fine to
share with anyone.

Class periods will not be recorded by me, nor are students permitted to
record the class lecture with any form of video, audio, or transcription
equipment. I will provide a separate recording of the lecture(s), as
well as accompanying slides, for any given class meeting \emph{after}
the class meeting it compliments. These materials are provided to help
students review the lecture content, if you believe there is something
you may have missed during the class period. \textbf{The posted lecture
recording and slides are NOT a substitute for being in class at the
designated meeting time - as you will miss important in-class
discussions and activities, if you are not present.}

\subsection{Attendance}\label{attendance}

Please try to make it to class, on time, for all scheduled class
periods. Coming in late may disrupt your peers' learning and my ability
to lecture. It is difficult to keep up with the course content if you
miss part of or all of the class. While I understand that life
circumstances may force you to miss class or be late on occasion, it
will likely impair your ability to succeed in this course. If you
suspect attendance will be consistently difficult for you in this
course, I would suggest that you consider leaving this section. If you
expect to miss a class period, or do so due to an emergency, please
reach out to me to make up work (see more details in
\hyperref[due-dates]{Due Dates}).

\subsection{Due Dates}\label{due-dates}

\textbf{All due dates, as they are listed in the schedule, are firm.}
You should closely track your own progress in the course and ensure that
you are keeping on pace. You should plan to make every attempt to attend
our regularly scheduled course times so that you may complete any
necessary assessments. Take-home assignments are due at the start of
class time.

If you expect that a qualifying pre-existing obligation, that cannot be
moved, will interfere with class time, please reach out to me \textbf{at
least one week prior to the affected course meeting.} So that we may
make arrangements for you to turn in relevant work, and take any missed
\hyperref[quizzes]{Quizzes} or \hyperref[exams]{Exams}. If you find
yourself in an emergency situation that prevents your attendance in
class/turning in work on time, please notify me as soon as possible and
provide documentation on what your emergency was (e.g., doctor's note).

I will work with students to ensure you have the opportunity to submit
your work and make it up. \textbf{Barring extraordinary circumstances,
you must reach out to me within 48 hours of the original due date to
coordinate an appropriate time to submit the work.}

In the cases of missed work needing to be made up, I ask that we
complete the work prior to the next class period as to not delay review
of exams or quizzes for your peers. \textbf{It is the student's
responsibility to contact me regarding make-up work - failure to do so
in a timely manner (48 hrs) will result in the total forfeiture of
points.}

Barring exemptions, as explained above, late work will have 25\% of
earned points deducted per each day late (including weekends). Work is
considered to be a day late immediately after submission for assignments
at the start of the class period. Example: a paper (50 points total) is
submitted 2 days late, and earns 40/50 points - this paper will have
50\% of earned points deducted, resulting in a 20/50. Thus, after 4 days
late, an assignment will not be eligible for any points.

Please submit all assignments through their respective
\hyperref[blackboard-ultra]{Blackboard Ultra} submissions, regardless of
whether they are late or not. Do not email me or hand in paper copies of
assignments unless I request so - I will redirect you to submit through
Blackboard.

\subsection{Changes to the Syllabus and
Schedule}\label{changes-to-the-syllabus-and-schedule}

It is plausible that events during the semester may require that I
modify the syllabus or schedule. As soon as I know such a change must be
made, I will notify you all through Blackboard and those messages should
be understood as amending this document.

\section{Blackboard Ultra}\label{blackboard-ultra}

I will be making liberal use of Blackboard Ultra to communicate,
administer assignments, and share course materials. Please make sure you
have a working knowledge of this platform and can reasonably navigate it
to find messages and submit assignments. You should have access to an
e-learning Blackboard tutorial course that can show you the basic
functions. I will try to use Blackboard to keep you all engaged during
the weeks and provide additional resources to those who are interested.
Prior to the course starting, please browse around and make sure you can
access and view everything alright. I will attempt to keep information
between this syllabus and schedule, and Blackboard as congruent as
possible. If you experience technical problems with Blackboard or are
not able to get things working in your browser, contact the help desk:

\textbf{Website:}
\url{https://www.gvsu.edu/elearn/help/blackboard-student-help-2.htm}

\textbf{Email:}
\href{mailto:helpdesk@gvsu.edu}{\nolinkurl{helpdesk@gvsu.edu}}

\textbf{Phone:} (616) 331-3513

If you find issues in due dates, grades, assignment format, etc., that
appear to be a mistake by me, please contact me and let me know, so I
can promptly fix it.

\section{Grading}\label{grading}

Summative assessment (i.e., graded assignments) are key to evaluating a
student's success in learning the content of this course. I have set up
this course to be forgiving for the sake of learning, as mistakes and
missteps are often part of the growth process. However, I must fairly
evaluate the abilities each student builds in this course, and must
ensure that the learning objectives are met. That being said, I care
deeply about the effort and motivation that you have put into this
course and have included assignments that award points for good-faith
attempts and offer room for improvement. My goal is to readily submit
grades on \hyperref[blackboard-ultra]{Blackboard Ultra} as I complete
them, so that you may be well-informed to your current status. All grade
functions on Blackboard should be fully set-up so that you can easily
track your progress and success.

\textbf{I cannot and will not ``bump'', ``curve'', ``round'', or
``scale'' grades at the individual student level, due to any
circumstance. Please do not ask me to do this.} Such a change in
student's grades can lead to unfairness and subjectivity in how students
are assessed, and would be unfair to your peers. In the case of an error
in grading or poor class performance on a test question, I will consider
the need to drop questions or adjust grading for all students equally.
The standard for what constitutes certain letter grades (e.g., A, A-, B,
etc.) in this course will be presented in this syllabus, and is subject
to (downward) change if I feel it is necessary.

I will provide an accurate breakdown of the sources of points, and
students are expected to monitor their own success throughout the term.
If your performance is below what you would like at any point, please
talk to me (see \hyperref[contacting-me]{Contacting Me}) or make use of
the \hyperref[additional-resources]{Additional Resources}. Please make
early adjustments to avoid any last minute issues that prevent you from
obtaining the success you'd like.

It is to the advantage of you, the student, that many points are
possible in this course. That way, a few poor performances can be made
up by consistent effort on the other available opportunities. The point
ranges are approximations of the percents - please note your final
letter grade will be determined solely by the percent range it falls in.
If I remove quiz or exam questions or change the point total at any
point, the point ranges will no longer properly apply.

\begin{longtable}[]{@{}lll@{}}
\toprule\noalign{}
Letter Grade & Percent & Points \\
\midrule\noalign{}
\endhead
\bottomrule\noalign{}
\endlastfoot
A & \textgreater93 & \textgreater474.3 \\
A- & 90 - 92.99 & 459 - 474.2 \\
B+ & 87 - 89.99 & 443.7 - 458.9 \\
B & 83 - 86.99 & 423.3 - 443.6 \\
B- & 80 - 82.99 & 408 - 423.2 \\
C+ & 77 - 79.99 & 392.7 - 407.9 \\
C & 73 - 76.99 & 372.3 - 392.6 \\
C- & 70 - 72.99 & 357 - 372.2 \\
D+ & 67 - 69.99 & 341.7 - 356.9 \\
D & 63 - 66.99 & 321.3 - 341.6 \\
F & \textless62.99 & \textless321.3 \\
\end{longtable}

\begin{longtable}[]{@{}lll@{}}
\toprule\noalign{}
Assignment & Points & Percentage \\
\midrule\noalign{}
\endhead
\bottomrule\noalign{}
\endlastfoot
Introductions & 10 & 2\% \\
Reading Evidence (5) & 10 x 5 = 50 & 9\% \\
Exams (2) & 100 x 2 = 200 & 40\% \\
Quizzes (12; 2 dropped) & 10 x 10 = 100 & 20\% \\
Research Proposal & 100 & 20\% \\
Article Critique & 50 & 9\% \\
Extra Credit (?) & - & - \\
------------------------ & --------------- & --------------- \\
Total & 510 & 100\% \\
\end{longtable}

\section{Activity Types}\label{activity-types}

I use a variety of activities as a source of points for your grade in
this course. This way, I can assess your skill in many domains and
mediums and can help you build a well-rounded skill set.

\subsection{Introductions}\label{introductions}

At the start of the semester, I ask that you all contribute to a
Blackboard discussion board, so your classmates and I may learn a little
bit about all of you. Please tell us:

\begin{itemize}
\tightlist
\item
  Your name (with phonemic spelling if you think it would help with
  pronunciation) and pronouns (optional - only if you feel comfortable)
\item
  Your year in school (e.g., First-year, sophomore, etc.)
\item
  Your majors and minors
\item
  Briefly identify one area of research in psychology you find
  interesting - it can be mundane or complex, broad or specific, just
  throw something out there.
\item
  Explain one or two interesting facts or hobbies of yours - share
  anything you'd like!
\end{itemize}

\subsection{Reading Evidence}\label{reading-evidence}

For any 5 chapter readings (out of 14) in the textbook, please submit
your notes that you take during your reading of the chapter. Your notes
can be handwritten or typed, and can be in whatever format and style
best suits your needs as a student. I just ask that you are
comprehensive and thorough to the content of the chapter and cover all
the sections to some degree. I may make suggestions on areas for
improvement, but will generally award full points if you make a good
faith attempt to cover all the content. It is recommended to do at least
some of these during early chapters, so I may suggest strategies that I
think will serve you well for the remainder of the semester. I do this
so that you may be set up for success in note-taking to best serve you
in this course, while also rewarding your dialogue with the reading.

Of course, I \textbf{strongly} recommend that you take good notes for
all chapters, not just the 5 ones you submit. They may aid you greatly
in the open-note quizzes and preparation on the exams.

\subsection{Exams}\label{exams}

There will be 2 exams in this course. These exams are intended to be
cumulative and will cover content from all prior units and
\hyperref[quizzes]{Quizzes}. Much of the knowledge in this course is
naturally cumulative, so it benefits you to review content from the
previous units. However, the majority of test content will come from the
most recent unit, with fewer questions being dedicated to the prior
unit. The format is as follows:

\begin{itemize}
\tightlist
\item
  Each exam is 50 multiple-choice questions, 2 points for each question
\item
  Exams will be taken at the start of the class period, but after the
  quiz review for the prior week. They will be taken on paper scantron
  sheets.
\item
  Exams will contain content from the entire unit, between lectures AND
  readings
\item
  Exams are timed, 113 minutes only (previously was 75 minutes)
\item
  Exams are not open-note, open-book, or collaborative. You are not
  permitted to use any form of assistance to aid you during the tests.
  Do not discuss the tests with other students after it has concluded
\item
  Quizzes and exam will be ended early if all students are clearly
  finished and content with their answers
\item
  Exams will be graded promptly and reviewed the following week
\end{itemize}

\subsection{Quizzes}\label{quizzes}

We will be taking quizzes routinely throughout the semester to help
cement the concepts between each class period. For each student, I will
drop each student's lowest 2 quiz grades from your final grade. The
format is as follows:

\begin{itemize}
\tightlist
\item
  Each quiz is 10 multiple-choice questions, 1 point for each question
\item
  Quizzes will be taken at the start of the class period on the
  Blackboard LMS
\item
  Quizzes will be on content covered in the previous lecture and the
  associated reading for that lecture
\item
  Quizzes are timed, 23 minutes only (previously was 15 minutes)
\item
  Quizzes are open-note and open-book, that is, you are allowed to use
  those resources during the quizzes. Thus, they reward good structure
  in thoughtfulness in your notes and preparation (see
  \hyperref[stay-organized]{Stay Organized})
\item
  You may not collaborate with others during the quizzes, or discuss
  questions with other students after the quiz. You cannot use AI tools
  or the internet to help you.
\item
  Quizzes and exam will be ended early if all students are clearly
  finished and content with their answers
\item
  Quizzes will be graded promptly and reviewed the following week
\end{itemize}

Because of the fast nature of quizzes, you will not have time to look up
answers to each of questions - please prepare by studying, reviewing,
and understanding the content, using your organized notes and book as a
quick backup.

\subsection{Research Proposal}\label{research-proposal}

This will be one of the most critical and important assignments in the
course. You will be asked to generate an original research proposal that
is ethical, well-designed, and rooted in a mature understanding of the
scientific literature. This project will use many of the skills that you
build throughout this course and should reflect a strong understanding
of your capabilities as a budding researcher. We will discuss this
project more after the first few weeks and I will provide a rubric
highlighting my grading standards for this project.

I will require that you identify a general topic for this proposal by
Oct 29, and that you submit a draft of this proposal by Nov 12.

\subsection{Article Critique}\label{article-critique}

You will be tasked with independently finding a piece of empirical,
peer-reviewed research in psychology that is interesting and accessible
to you. Then I will ask that you critically assess the entire research
piece to the best of your ability, critiquing the ethics, validity,
design, and conclusions of the study. The full assignment will be
discussed more in-depth later in the semester, as it is most appropriate
to begin this project after you have demonstrated skills in literature
review. However, you may want to begin casually searching for an
interesting article early on, as I will ask you get your choice approved
by me. I will provide a rubric highlighting my grading standards for
this project.

I will require that you are able to locate and provide me with the
APA-style citation and full-text PDF of your chosen article. Your choice
of article will need to be conveyed to me by Oct 29, and a draft will
need to be submitted by Nov 12. Before starting this, you will want to
be more familiar with many of the concepts present in the lectures.
Later in the semester, we will do in-class article discussions to help
prepare you for writing this critique.

\subsection{Extra Credit (?)}\label{extra-credit}

At this time, I do not plan on offering any extra credit for this
course. \textbf{Please do not ask me to offer it and do not plan on it
being offered.} \emph{If} I find an opportunity that I feel will enhance
your experience in this course and be eligible for extra credit, I will
notify you all through Blackboard. \emph{If} I do add extra credit
assignments, they will supplement the existing points total, rather than
adding to the total itself. This means that they will functionally
``make up'' for lost points on the other assignments.

\section{Schedule}\label{schedule}

This schedule will be a rough plan for the semester. In addition to each
scheduled quiz/exam and in-class activity, keep in mind I will lecture
on the assigned reading for that week. There is the possibility that
this schedule will need to change if lectures take longer than expected
or if classes are canceled unexpectedly. I will communicate if deadlines
change.

\begin{longtable}[]{@{}
  >{\raggedright\arraybackslash}p{(\columnwidth - 8\tabcolsep) * \real{0.1446}}
  >{\raggedright\arraybackslash}p{(\columnwidth - 8\tabcolsep) * \real{0.1205}}
  >{\raggedright\arraybackslash}p{(\columnwidth - 8\tabcolsep) * \real{0.1928}}
  >{\raggedright\arraybackslash}p{(\columnwidth - 8\tabcolsep) * \real{0.2651}}
  >{\raggedright\arraybackslash}p{(\columnwidth - 8\tabcolsep) * \real{0.2771}}@{}}
\toprule\noalign{}
\begin{minipage}[b]{\linewidth}\raggedright
Class Date
\end{minipage} & \begin{minipage}[b]{\linewidth}\raggedright
Reading
\end{minipage} & \begin{minipage}[b]{\linewidth}\raggedright
Quiz / Exam
\end{minipage} & \begin{minipage}[b]{\linewidth}\raggedright
In-class Activity
\end{minipage} & \begin{minipage}[b]{\linewidth}\raggedright
Due
\end{minipage} \\
\midrule\noalign{}
\endhead
\bottomrule\noalign{}
\endlastfoot
Aug 27 & Ch 1 & - & APA Style Practice & - \\
Sep 3 & Ch 2 & Quiz 1 & Lit Search Demo 1 & Introductions \\
Sep 10 & Ch 3 & Quiz 2 & Lit Search Demo 2 & Reading Evidence I \\
Sep 17 & Ch 4 & Quiz 3 & Ethics Case Study & - \\
Sep 24 & Ch 5 & Quiz 4 & Measures Search & - \\
Oct 1 & Ch 6 & Quiz 5 & Exam 1 Study Guide & - \\
Oct 8 & Ch 7 & Exam 1 & Writing Prep & - \\
Oct 15 & Ch 8 & Quiz 7 & Article Discussion & - \\
Oct 22 & - & - & - & - \\
Oct 29 & Ch 9 & Quiz 8 & Writing Time & Writing Topics \\
Nov 5 & Ch 10 & Quiz 9 & Article Discussion & - \\
Nov 12 & Ch 11 & Quiz 10 & Writing Time & Writing Drafts \\
Nov 19 & Ch 12 & Quiz 11 & TBD & - \\
Nov 26 & Ch 13 & Quiz 12 & TBD & Reading Evidence II-V \\
Dec 3 & Ch 14 & Quiz 13 & Exam 2 Study Guide & Writing Finals \\
Dec 7 - 14 & - & Exam 2 / Final & - & - \\
\end{longtable}

\textbf{Notes on Reading the Schedule:}

\begin{itemize}
\item
  Quiz content always corresponds to the reading and lecture from the
  \emph{prior} class meeting, e.g., Quiz 7 will cover content from Ch 7
  reading and lecture content from Oct 8. The week after a quiz, we will
  review problematic concepts from that quiz.
\item
  ``Writing'' is shorthand that represents the article critique AND
  research proposal projects
\item
  Chapters 6 (Oct 1) and 14 (Dec 3) and their corresponding lectures, do
  not have a quiz associated with them, but will be covered by the
  exams.
\item
  Chapter 7 and it's associated lecture (on Oct 8) will NOT be covered
  by exam 1
\item
  The timing of the course final (exam 2) will be decided by the school
  policy. I will communicate what the exact date and time of the exam
  will be later on.
\end{itemize}

\textbf{Other Important Dates:}
\url{https://www.gvsu.edu/registrar/academiccalendar.htm\#5CEEFECB-98D7-18B7-8BE44596135F45D8}

\section{Additional Resources}\label{additional-resources}

Outside myself, the textbook, and course content, there is a wealth of
student resources available to you all. While I will always aim to be
helpful and available to the best of my ability, I strongly encourage
you to utilize these resources to support your learning and
accommodation. Unfortunately, I am unable to offer extensive,
individualized help to each student and may refer you to one of these
services should you request that. Please visit the respective webpages
for these services to learn more about what they can offer to you. There
are many more than I can list, and you should definitely look at all the
available services here:
\url{https://www.gvsu.edu/clas/supporting-students-1136.htm}

\subsection{Disability Accommodations}\label{disability-accommodations}

\textbf{Webpage:} \url{http://www.gvsu.edu/dss}

Any student who requires accommodation because of a physical or learning
disability must contact Disability Support Services (DSS;
\url{http://www.gvsu.edu/dss}) at 616-331-2490 as soon as possible. It
is the student's responsibility to request assistance from DSS. After
you have documented your disability, please contact me to set up an
appointment or see me during office hours to discuss your specific needs
in accordance with your documentation.

Accommodations are always designed to maintain the academic integrity of
the course; student with disabilities are held to the same academic
standards as all other students. Accordingly, if no additional costs
(including staff time) are involved, the instructor will extend such
accommodation to anyone who requests them, whether or not the student
has a declared disability. If a requested accommodation requires special
equipment, space, personnel, staff time, or other resources beyond those
normally available to the class, the accommodation will be offered only
if the student has gone through the process that begins by declaring the
disability with Disability Support Services. If you have a disability
and think you might need assistance evacuating the classroom and/or
building in an emergency situation, please make sure I am aware, so I
can develop a plan to assist you.

Please know that I am committed to making my classroom friendly and
accessible to students of all needs and backgrounds. If there is
something minor I can do to my presentations and materials to make them
more accessible to you, please do let me know.

\subsection{The Tutoring and Reading Center (One-on-one
Tutoring)}\label{the-tutoring-and-reading-center-one-on-one-tutoring}

\textbf{Webpage:} \url{https://www.gvsu.edu/tutoring/}

\textbf{Description}

Tutoring in the College of Liberal Arts and Sciences (CLAS) serves
students at Grand Valley State University by providing tutoring and
supplemental instruction. We foster academic success by providing
multidisciplinary content support and promoting positive study behaviors
to cultivate empowered, persistent learners in an inclusive, accessible,
and learner-centered environment.

\subsection{Fred Meijer Center for Writing and Michigan Authors (Writing
Help)}\label{fred-meijer-center-for-writing-and-michigan-authors-writing-help}

\textbf{Webpage:} \url{https://www.gvsu.edu/wc}

\textbf{Description}

The Fred Meijer Center for Writing provides writing assistance to all
GVSU students, on any type of project and at any stage of the process.
The Writing Center employs both undergraduate and graduate writing
consultants from across majors and disciplines. Consultants are trained
to help writers brainstorm, organize, or develop their ideas; and they
can help writers edit their own work and document sources correctly. The
Center's services are free and students can work with an idea,
assignment prompt, or draft of their paper. Students can virtually drop
in or schedule an appointment; both appointments and drop-ins are
available during all service hours: (Mon-Thurs 9am-11pm, Friday 9am-3pm,
Sunday 2pm-11pm). Due to COVID-19, all writing center services are
available online. Limited in-person consulting may be available; please
check the Writing Center's website for up-to-date information. All
service options (drop- ins, appointments, email support) can be accessed
via the Writing Center's online scheduling system - Book It. We look
forward to working with you!

\subsection{GVSU Knowledge Market (Research, Writing, and Presentations
Help)}\label{gvsu-knowledge-market-research-writing-and-presentations-help}

\textbf{Webpage:} \url{https://www.gvsu.edu/library/km/}

\textbf{Description}

The Knowledge Market is an interdisciplinary peer-to-peer collaborative
service that brings together similarly-aligned academic programs to help
students develop their intellectual skills, habits, and identities. The
Knowledge Market offers one-stop support for library research, writing,
oral presentations, and digital projects! Available to help with
projects from any class, our highly-trained consultants are here for
you.

We have two convenient locations and offer online sessions, so you can
meet with us from wherever you are. To have a consultation, schedule an
appointment or join the drop-in queue at any of our locations. We look
forward to working with you!

\subsection{Psych Friends Peer Mentoring (Peer-to-peer
Mentoring)}\label{psych-friends-peer-mentoring-peer-to-peer-mentoring}

\textbf{Webpage:}
\url{https://www.gvsu.edu/psychology/psych-friends-477.htm}

Psych Friends mentors are upper-level psychology and behavioral
neuroscience students who are trained to provide support in many areas,
such as: effective studying and time management techniques, exam
preparation and reflection skills, understanding the PSY and BNS major
requirements, potential jobs and careers in the field, the process of
applying for graduate school, and how to maintain physical and mental
health as a student. Visit \url{https://www.gvsu.edu/navigate} to
schedule an online or in-person meeting today!

Psych Friends Peer Mentoring aims to increase the academic success \&
well-being of psychology students by connecting upper division
psychology majors (mentors) with students newer to the major (mentees).
Psych Friends is a great way to expand your knowledge on all that the
field of psychology has to offer and to build a social network with
others within the psychology community.

Make an appointment with a peer mentor to learn about the psychology and
behavioral neuroscience major requirements, careers, and education paths
in psychology, the graduate school application process, strategies to
increase academic success, and/ or student self-care techniques. Our
mentors would love to get to know you. We have plenty of time slots
available to best fit your needs. Appointments can be made through
Navigate and students can choose between having an appointment in person
on campus, or to meet online via Zoom.

\subsection{COVID-19 Resources}\label{covid-19-resources}

\textbf{Webpage:} \url{https://www.gvsu.edu/lakerstogether/}

Unfortunately, COVID-19 remains a threat to our student community.
Please review the above link for resources to help protect yourself and
others.

\section{FAQ}\label{faq}

\textbf{How do you say the professor's name and how should I address
you?}

The phonemic spelling is Kwin-tin Kwahg-lee-ah-no. You are welcome to
call me Quinton, Prof.~Quagliano, or Prof.~Q. Please do not address me
as Mr.~Quagliano (I find it odd) or Dr.~Quagliano (I have not earned
that honorific).

\textbf{What is the professor's background?}

I have an B.S. in Psychology (neuroscience concentration) from
\href{https://calvin.edu/}{Calvin University}, and an M.S. in
Quantitative Psychology from \href{https://www.bsu.edu/}{Ball State
University}. I am a full-time
\href{https://www.napnet.org/what}{psychometrist} at Trinity Health
Grand Rapids in addition to my adjunct role here. Generally, I describe
myself as being research-oriented with emphasis on advanced quantitative
(statistical) methods. I have had the pleasure on working on projects in
a variety of theoretical domains, such as education, neuropsychology,
neurology, psychiatry, and audiology. I remain loosely associated with
Pine Rest Christian Mental Health Services, Ball State University, and
Trinity Health Grand Rapids Neurology/Neuropsychology for ongoing
research projects.

I was a TA and tutor for students throughout my undergraduate and
graduate education, and have been involved in clinical work (inpatient
psychiatry and outpatient neuropsychology) for the last 6 years. My CV
will be posted in the course Blackboard if you'd like to know more about
me.

\textbf{Can I talk to the professor about things other than class?}

Of course! I love talking about research, careers, your academic
journey, and your personal goals. You can always email me or stop by
during office hours for a chat. Unfortunately, I cannot offer any
research, teaching assistant (TA), or extracurricular opportunities to
students at this time, but I can give you pointers on where to look for
these things if you are interested! I can serve as a reference or
``letter-writer'' for your future endeavors, but only if you ask me.

\textbf{Why so many quizzes?}

More assessments mean that a few bad scores won't sink your chances of a
higher letter grade. Constant testing and revisiting of past topics are
also excellent ways to learn and solidify concepts. It also lets me know
if the class is struggling with certain topics, so I have time to
re-hash ideas before the more-important exams. I know it may be
stressful to have these every meeting, but I do promise they have a
purpose to them. Take good notes revisit topics often and you should be
okay.

\textbf{How can I learn to write better?}

One of the best ways to learn how to scholarly write is to constantly
read published, scientific articles. I've refined my own writing style a
lot by just emulating the style of more experienced scientists. Of
course, good writing is a skill build up over the course of many years;
so be patient and consistent in developing your abilities. I am here to
help you prepare your written assignments, but make sure to use the
writing center and other resources as well!

\textbf{Should I drop the course?}

I cannot make that decision for you, but I do encourage you to be
constantly reflecting upon your success in the course. I try to provide
timely feedback and review, so you know how you are performing on
things. Sometimes, it is better to drop a course and return when you are
more experienced, and there is no shame in doing so. I would suggest you
speak with your academic adviser prior to making this decision, so that
they can explain any ramifications on your overall degree progress.

\textbf{Can I re-take or re-submit an assignment, exam, or quiz for a
revised score?}

No.~I give you adequate time, resources, and in-class support to study
for quizzes/exams and prepare assignments. I will give constant feedback
so that you are better equipped for future assignments, but cannot
re-grade revised assignments. I'd like for students to be proactive in
creating high-quality work the first time by starting work early and
using good study habits.

\textbf{Why are the article critique and research proposal due before
the final exam?}

My hope is that, by requiring these due prior to the final exams, you
can better focus your efforts at the end of the semester towards
studying and preparing for the exam in this course, and your other
courses. It also means I can grade these quickly and have your course
grade fully updated going into the final. This way, you know where you
are at and can get a sense of your ability to aim for a certain letter
grade. Because I have due dates for your topics and early drafts of both
of these projects, you likely will already have a good start on both
ahead of the due dates.

\section{Tips for Success}\label{tips-for-success}

\subsection{Back Up Your Work}\label{back-up-your-work}

We have all been in the undesirable position of losing hours of work due
to a sudden power outage or accident. I \textbf{strongly} recommend that
you use platforms and methods to maintain your work in case of an
accident. Use services like Microsoft OneDrive/365 or Google Drive to
save your work in the cloud, or use software like git to version control
your work. Save regularly and often, enabling auto-save if it is a
feature in the software you use. While I empathize with the pain of
suddenly losing work, technology issues are not an acceptable excuse for
late work - as there are a variety of ways to preemptively protect your
progress.

\subsection{Stay Organized}\label{stay-organized}

The start of the semester is the best time to establish the right
organization method that keeps you on track. Losing assignments, papers,
and notes to the void can feel like a huge setback. Think critically now
about how you want to organize your physical/digital documents, and get
it ready now. Stick with it, and you'll find it much easier to look back
later when you need that knowledge! But remember to
\hyperref[back-up-your-work]{Back Up Your Work}!

\subsection{Create a Weekly Schedule}\label{create-a-weekly-schedule}

While I provide a course \hyperref[schedule]{Schedule} to give structure
to the overall class, I would recommend setting up a personal schedule
for studying and attending to assignments throughout each week.
Procrastinating and trying to rapidly complete work before deadlines
will hurt the quality of your work and that will likely be reflected in
your grade. Establishing good time-management habits early on in the
semester will help you maintain a good balance between your life and
classes.

\subsection{Communicate Well With Me}\label{communicate-well-with-me}

I ask that you regularly attend to your email and Blackboard so that you
see important notifications from me regarding this course. If you need
help or have question please do reach out to me as soon as possible
(preferably via email). Ignorance or avoidance of emails and
notification is not an excuse for late or poor work. I am always happy
to point you in the right direction or clarify something, but I can only
do so if you tell me your concerns.

\subsection{Take Care of Yourself}\label{take-care-of-yourself}

Balancing undergraduate studies with the stresses of life is difficult
and is a time of change for many students. Make sure you get enough
sleep, eat food and drink water, and spend time with friends and family.
You'll do best in my course (and every course) when you are at your best
mentally and emotionally! I look forward to supporting you all best I
can this term!

\emph{This course is subject to the GVSU policies listed at
\url{http://www.gvsu.edu/coursepolicies/}}




\end{document}
