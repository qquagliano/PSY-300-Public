% Options for packages loaded elsewhere
\PassOptionsToPackage{unicode}{hyperref}
\PassOptionsToPackage{hyphens}{url}
\PassOptionsToPackage{dvipsnames,svgnames,x11names}{xcolor}
%
\documentclass[
  12pt,
  letterpaper,
]{scrartcl}

\usepackage{amsmath,amssymb}
\usepackage{iftex}
\ifPDFTeX
  \usepackage[T1]{fontenc}
  \usepackage[utf8]{inputenc}
  \usepackage{textcomp} % provide euro and other symbols
\else % if luatex or xetex
  \usepackage{unicode-math}
  \defaultfontfeatures{Scale=MatchLowercase}
  \defaultfontfeatures[\rmfamily]{Ligatures=TeX,Scale=1}
\fi
\usepackage{lmodern}
\ifPDFTeX\else  
    % xetex/luatex font selection
\fi
% Use upquote if available, for straight quotes in verbatim environments
\IfFileExists{upquote.sty}{\usepackage{upquote}}{}
\IfFileExists{microtype.sty}{% use microtype if available
  \usepackage[protrusion = true]{microtype}
  \UseMicrotypeSet[protrusion]{basicmath} % disable protrusion for tt fonts
}{}
\makeatletter
\@ifundefined{KOMAClassName}{% if non-KOMA class
  \IfFileExists{parskip.sty}{%
    \usepackage{parskip}
  }{% else
    \setlength{\parindent}{0pt}
    \setlength{\parskip}{6pt plus 2pt minus 1pt}}
}{% if KOMA class
  \KOMAoptions{parskip=half}}
\makeatother
\usepackage{xcolor}
\usepackage[margin = 1in]{geometry}
\setlength{\emergencystretch}{3em} % prevent overfull lines
\setcounter{secnumdepth}{2}
% Make \paragraph and \subparagraph free-standing
\makeatletter
\ifx\paragraph\undefined\else
  \let\oldparagraph\paragraph
  \renewcommand{\paragraph}{
    \@ifstar
      \xxxParagraphStar
      \xxxParagraphNoStar
  }
  \newcommand{\xxxParagraphStar}[1]{\oldparagraph*{#1}\mbox{}}
  \newcommand{\xxxParagraphNoStar}[1]{\oldparagraph{#1}\mbox{}}
\fi
\ifx\subparagraph\undefined\else
  \let\oldsubparagraph\subparagraph
  \renewcommand{\subparagraph}{
    \@ifstar
      \xxxSubParagraphStar
      \xxxSubParagraphNoStar
  }
  \newcommand{\xxxSubParagraphStar}[1]{\oldsubparagraph*{#1}\mbox{}}
  \newcommand{\xxxSubParagraphNoStar}[1]{\oldsubparagraph{#1}\mbox{}}
\fi
\makeatother

\providecommand{\tightlist}{%
  \setlength{\itemsep}{0pt}\setlength{\parskip}{0pt}}\usepackage{longtable,booktabs,array}
\usepackage{calc} % for calculating minipage widths
% Correct order of tables after \paragraph or \subparagraph
\usepackage{etoolbox}
\makeatletter
\patchcmd\longtable{\par}{\if@noskipsec\mbox{}\fi\par}{}{}
\makeatother
% Allow footnotes in longtable head/foot
\IfFileExists{footnotehyper.sty}{\usepackage{footnotehyper}}{\usepackage{footnote}}
\makesavenoteenv{longtable}
\usepackage{graphicx}
\makeatletter
\def\maxwidth{\ifdim\Gin@nat@width>\linewidth\linewidth\else\Gin@nat@width\fi}
\def\maxheight{\ifdim\Gin@nat@height>\textheight\textheight\else\Gin@nat@height\fi}
\makeatother
% Scale images if necessary, so that they will not overflow the page
% margins by default, and it is still possible to overwrite the defaults
% using explicit options in \includegraphics[width, height, ...]{}
\setkeys{Gin}{width=\maxwidth,height=\maxheight,keepaspectratio}
% Set default figure placement to htbp
\makeatletter
\def\fps@figure{htbp}
\makeatother

% NOTE: KOMAscript settings for header and footer
\usepackage[headsepline = 1.5pt, footsepline = 1.5pt]{scrlayer-scrpage}
\setkomafont{pagenumber}{\bfseries}
\setkomafont{pagehead}{\bfseries}
\setkomafont{pagefoot}{\itshape}
\lohead*{Research Proposal Instructions and Rubric}
\rohead*{\pagemark}
\cofoot*{"Scientists have become the bearers of the torch of discovery in our quest for knowledge" - Stephen Hawking}

% NOTE: Color settings
\usepackage{color}
\definecolor{gvblue}{HTML}{0032A0} % NOTE: Setting a new color
\renewcommand\thesection{\color{gvblue}\arabic{section}} % NOTE: Change section number color

% NOTE: Symbols, Fonts, and typesetting
\usepackage{bbding} % NOTE: For checkmark with \Checkmark
\usepackage{mathtools, amsthm, amssymb, amsfonts}
\usepackage{lmodern}
\usepackage[utf8]{inputenc}

% NOTE: Double-spacing and indents
% \usepackage{setspace} % NOTE: Allows for specifying spaces
% \usepackage{indentfirst} % NOTE: Indent first paragraph

\usepackage{multicol}
\makeatletter
\@ifpackageloaded{caption}{}{\usepackage{caption}}
\AtBeginDocument{%
\ifdefined\contentsname
  \renewcommand*\contentsname{Table of contents}
\else
  \newcommand\contentsname{Table of contents}
\fi
\ifdefined\listfigurename
  \renewcommand*\listfigurename{List of Figures}
\else
  \newcommand\listfigurename{List of Figures}
\fi
\ifdefined\listtablename
  \renewcommand*\listtablename{List of Tables}
\else
  \newcommand\listtablename{List of Tables}
\fi
\ifdefined\figurename
  \renewcommand*\figurename{Figure}
\else
  \newcommand\figurename{Figure}
\fi
\ifdefined\tablename
  \renewcommand*\tablename{Table}
\else
  \newcommand\tablename{Table}
\fi
}
\@ifpackageloaded{float}{}{\usepackage{float}}
\floatstyle{ruled}
\@ifundefined{c@chapter}{\newfloat{codelisting}{h}{lop}}{\newfloat{codelisting}{h}{lop}[chapter]}
\floatname{codelisting}{Listing}
\newcommand*\listoflistings{\listof{codelisting}{List of Listings}}
\makeatother
\makeatletter
\makeatother
\makeatletter
\@ifpackageloaded{caption}{}{\usepackage{caption}}
\@ifpackageloaded{subcaption}{}{\usepackage{subcaption}}
\makeatother

\usepackage{hyphenat}
\usepackage{ifthen}
\usepackage{calc}
\usepackage{calculator}



\usepackage{graphicx}
\usepackage{geometry}
\usepackage{afterpage}
\usepackage{tikz}
\usetikzlibrary{calc}
\usetikzlibrary{fadings}
\usepackage[pagecolor=none]{pagecolor}


% Set the titlepage font families







% Set the coverpage font families


\ifLuaTeX
\usepackage[bidi=basic]{babel}
\else
\usepackage[bidi=default]{babel}
\fi
\babelprovide[main,import]{american}
% get rid of language-specific shorthands (see #6817):
\let\LanguageShortHands\languageshorthands
\def\languageshorthands#1{}
\ifLuaTeX
  \usepackage{selnolig}  % disable illegal ligatures
\fi
\usepackage{bookmark}

\IfFileExists{xurl.sty}{\usepackage{xurl}}{} % add URL line breaks if available
\urlstyle{same} % disable monospaced font for URLs
\hypersetup{
  pdftitle={PSY-300 Research Proposal Instructions and Rubric},
  pdfauthor={Quinton Quagliano, M.S.},
  pdflang={en-US},
  pdfsubject={PSY-300 Research Proposal Instructions and Rubric},
  pdfkeywords={syllabus},
  colorlinks=true,
  linkcolor={gvblue},
  filecolor={Maroon},
  citecolor={gvblue},
  urlcolor={gvblue},
  pdfcreator={LaTeX via pandoc}}


\title{Research Proposal Instructions and Rubric - Fall 2024}
\usepackage{etoolbox}
\makeatletter
\providecommand{\subtitle}[1]{% add subtitle to \maketitle
  \apptocmd{\@title}{\par {\large #1 \par}}{}{}
}
\makeatother
\subtitle{PSY-300: Research Methods in Psychology}
\author{Quinton Quagliano, M.S.}
\date{Friday, October 11, 2024}

\begin{document}
%%%%% begin titlepage extension code


\begin{titlepage}

%%% TITLE PAGE START

% Set up alignment commands
%Page
\newcommand{\titlepagepagealign}{
\ifthenelse{\equal{center}{right}}{\raggedleft}{}
\ifthenelse{\equal{center}{center}}{\centering}{}
\ifthenelse{\equal{center}{left}}{\raggedright}{}
}


\newcommand{\titleandsubtitle}{
% Title and subtitle
{{\huge{\bfseries{\nohyphens{Research Proposal Instructions and Rubric -
Fall 2024}}}}\par
}%

\vspace{\betweentitlesubtitle}
{
{\Large{\nohyphens{PSY-300: Research Methods in Psychology}}}\par
}}
\newcommand{\titlepagetitleblock}{
\newcommand{\HRule}{\rule{\linewidth}{0.5mm}} 

\HRule\\[0.4cm]

\titleandsubtitle

\HRule\\
}
\newcommand{\authorstyle}[1]{{\small{#1}}}

\newcommand{\affiliationstyle}[1]{{\small{#1}}}

\newcommand{\titlepageauthorblock}{
\newlength{\miniA}
\setlength{\miniA}{0pt}
\newlength{\namelen}
\settowidth{\namelen}{Quinton Quagliano,
M.S.}\setlength{\miniA}{\maxof{\miniA}{\namelen}}
\setlength{\miniA}{\miniA+0.05\textwidth}
\newlength{\miniB}
\setlength{\miniB}{0.99\textwidth - \miniA}
\begin{minipage}{\miniA}
\begin{flushleft}
{\authorstyle{Quinton Quagliano, M.S.}}
\end{flushleft}
\end{minipage}
\begin{minipage}{\miniB}
\begin{flushright}
{\affiliationstyle{Department of Psychology
\\}}
\end{flushright}
\end{minipage}}

\newcommand{\titlepageaffiliationblock}{}
\newcommand{\headerstyled}{%
{\textsc{\LARGE{}}}
}
\newcommand{\footerstyled}{%
{}
}
\newcommand{\datestyled}{%
{\large{Friday, October 11, 2024}}
}


\newcommand{\titlepageheaderblock}{\headerstyled}

\newcommand{\titlepagefooterblock}{
\footerstyled
}

\newcommand{\titlepagedateblock}{
\datestyled
}

%set up blocks so user can specify order
\newcommand{\titleblock}{\newlength{\betweentitlesubtitle}
\setlength{\betweentitlesubtitle}{\baselineskip}
{

{\titlepagetitleblock}
}

\vspace{1.5cm}
}

\newcommand{\authorblock}{{\titlepageauthorblock}

\vspace{1.5cm}
}

\newcommand{\affiliationblock}{{\titlepageaffiliationblock}

\vspace{0pt}
}

\newcommand{\logoblock}{{\includegraphics[width=250pt]{gvlogo.png}}

\vspace{2\baselineskip}
}

\newcommand{\footerblock}{}

\newcommand{\dateblock}{{\titlepagedateblock}

\vspace{0pt}
}

\newcommand{\headerblock}{}

\thispagestyle{empty} % no page numbers on titlepages


\newlength{\minipagewidth}
\setlength{\minipagewidth}{\textwidth}
\raggedright % single minipage
% [position of box][box height][inner position]{width}
% [s] means stretch out vertically; assuming there is a vfill
\begin{minipage}[b][\textheight][s]{\minipagewidth}
\titlepagepagealign
\headerblock

\logoblock

\titleblock

\authorblock
\par

\end{minipage}\ifthenelse{\equal{}{right} \OR \equal{}{leftright} }{
\hspace{\B}
\vrulecode}{}
\clearpage
%%% TITLE PAGE END
\end{titlepage}
\setcounter{page}{1}

%%%%% end titlepage extension code

% NOTE: Better hold for LaTeX tables
\floatplacement{table}{H}

% NOTE: Set KOMAscript headings
\pagestyle{scrheadings}

% NOTE: Double-spacing and indents
% \doublespacing
\frenchspacing % NOTE: Only single space after period
% \setlength{\parindent}{20pt} % NOTE: Amount of spacing in indent

\renewcommand*\contentsname{Table of Contents}
{
\hypersetup{linkcolor=}
\setcounter{tocdepth}{2}
\tableofcontents
}

\newpage{}

\section{Assignment Instructions}\label{assignment-instructions}

This research proposal is the definitive take-home assignment for the
course and will require that you use many of the skills you have built
through the semester. This assignment allows me to best assess your
mastery of literature review, scientific writing, and sound research
design in a format which will prepare you well for future courses.
Whereas the exams and quizzes will test your knowledge, this assignment
will require a strong command of applying the concepts you've learned.
Please write in a manner that would allow this plan to be followed by
someone actually attempting to conduct this research.

In class, we will have workshops, tutorials, and guided writing time to
support you in this project. It would be very wise to start this paper
as early as possible and work on it steadily through the second half of
the semester. Robust research projects in the ``real world'' are often
planned over the course of weeks or months! Take your time, meet the
preliminary deadlines, and use all the resources available to you - and
you will do well on this assessment.

You may find it helpful to work on this alongside your article critique
project. Whereas that critique will focus on your ability to read and
recognize ideas in research, this project will put more emphasis on your
own ability to explain and rationalize certain design decisions.

The end product should contain these 4 sections:

\subsection{Introduction}\label{introduction}

A thorough literature review establishing context for your planned
study. You should use vetted, peer-reviewed sources that add support to
the topic you choose as a valuable and meaningful area of investigation.
Scaffold naturally from foundation concepts in the research area to the
``gap'' that you are trying to fill with the proposal. Effectively, you
are making an argument for why this research is needed and providing the
necessary information for a reader to understand your starting point.
You will also need to write a clear hypothesis that articulates the goal
of the study and expected outcome.

\subsection{Methods}\label{methods}

Layout a complete plan for sampling, measures, procedures, and planned
analyses for your study. Your plan should balance ethics, power, and
practicality in such a way that the study could feasibly be completed
with adequate time and resources. You should also reasonably address
each area of validity we have discussed in class and the impact of
design decisions on the strength and generalizability of claims. Where
necessary, consider citing previous work in the area with similar
methodology that buttresses your approach. Give detail on the scale of
measurement, and rationale your use of particular statistical tests that
are appropriate for these tests. (\emph{Do not worry too much about
in-depth statistical procedures, that will be covered further in future
course - but it will help to get familiar sooner})

\subsection{Discussion}\label{discussion}

A hypothetical discussion of what the implications are for whether
results are significant or not. Make sure to clearly connect your
``results'' back to the literature and ``gap'' you provide in your
introduction. Explain what your findings would mean in the context of
existing research, and provide appropriate and rational discussion of
limitations and future directions for research.

\subsection{References}\label{references}

An APA 7th Edition-style bibliography that contains the full information
for all citations used in the project. If you use tools such as Citation
Machine or Zotero/EndNote, make sure to double-check that citations are
correct, as sometimes automated tools can cause unexpected issues.

\section{Timeline of Project Steps and Due
Dates}\label{timeline-of-project-steps-and-due-dates}

Submission of each step on time is accounted for by points for the
project. Please make sure that you stay on track to earn all of your
points and so that you do not risk having an overwhelming amount of work
at the end of the semester.

\subsection{Topic Approval}\label{topic-approval}

You must submit a brief summary (3 - 4 sentences) of a topic of your
choice for my approval, and include a rough idea of what type of design
you could use. This study must be psychological in nature, but may be in
any domain of psychology. That being said, I would encourage you to
choose a topic that both has adequate amounts of literature about it AND
is personally interesting to you. You also want to ensure that the topic
can be investigated through the methods we have learned in this course.
You may have to do an early literature review to ensure there are enough
papers out there in order to write a proper introduction. I require that
you submit this description for approval by Oct 29. (5 pts)

\subsection{First Draft}\label{first-draft}

You will provide an early draft of your proposal, which allows me to
assess your progress and direction, and provide you with feedback. Part
of this draft will need to be an annotated bibliography with a minimum
of 10 peer-reviewed sources, published since 2000. An annotated
bibliography looks similar to what a ``References'' section looks like
in a paper, but with written notes for each citation explaining how it
will be used in the paper. I will expect you to at least have outlined
and/or began writing your introduction at this point. Properly tracking
and integrating sources is one of the most difficult parts about
planning research, so please put good time into reading and evaluating
articles for inclusion in your final project.

It is to your advantage to finish as much of your proposal as you can
before this deadline, as I can give more feedback, and it will set you
up better for an easier final submission. This is due by Nov 12. (15
pts)

\subsection{Final Draft}\label{final-draft}

A final submission of the completed project. This will, ideally, be a
natural continuation of your chosen topic and draft from the prior
parts. I expect this to be well-written and edited, in APA 7 style, use
adequate amounts of peer-reviewed citations, and flow well in its logic.
This is due by Dec 3. (80 pts)

\section{Rubric}\label{rubric}

There is no word or page limit or minimum, but please make sure to
include enough content to do justice to the topic and each section,
while still being clear and concise. If you have concerns about your
length, please reach out to me.

\begin{longtable}[]{@{}ll@{}}
\toprule\noalign{}
Points & Component \\
\midrule\noalign{}
\endhead
\bottomrule\noalign{}
\endlastfoot
5 & Topic Approval \\
15 & First Draft \\
10 & Final Draft - Writing Style \\
20 & Final Draft - Introduction \\
20 & Final Draft - Methods \\
20 & Final Draft - Discussion \\
10 & Final Draft - References \\
\end{longtable}

\section{Tips}\label{tips}

\subsection{Start Early}\label{start-early}

While you won't be able to start on this project until you have some
knowledge under your belt, I recommend you at least begin considering
options for this project early in the semester. I myself have been
guilty of procrastinating writing, and I know how much a pain it is to
have to make it up quickly. Make sure you spread your writing out and
revisit your work to re-edit and add content where needed.

\subsection{DO NOT USE AI}\label{do-not-use-ai}

While this is against my course policy in general (see the syllabus), I
can't emphasize enough that you MUST use your own words and work to
perform well. AI is notorious for making up citations that do not exist
and mis-citing information. In such a case, you would not just be guilty
of using a prohibited tool, you would also be plagiarizing. I want each
of you to succeed and grow on your own merits - and using compensatory
strategies will take away from your learning. If you feel you must use
it to aid your writing, please triple-check your work and citations.

\subsection{Read More Articles}\label{read-more-articles}

This seems intuitive, but really, reading well written scientific
articles will help give you an idea of what the flow of scientific
writing sounds like. The APA 7 Manual also has many helpful suggestions
for how to approach writing in this style. Research oftentimes reads
very different from other form of literature, so it is critical to
understand how it tends to read and provide information.

\subsection{Use ALL of Your Resources}\label{use-all-of-your-resources}

Come talk to me. Go to the librarians. Read up on research guides from
the library and college. Have the Writing Center help you edit your
early work. You have many opportunities to craft a stellar piece of work
- make sure you use them readily! There is no shame in asking for help,
but you must do so early when there is still time to correct things.




\end{document}
